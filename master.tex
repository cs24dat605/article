\documentclass[twocolumn]{article}
% Bibliography used for citing
\usepackage[backend=biber,
bibencoding=utf8,
style=numeric-comp
]{biblatex}
\addbibresource{bib/mybib.bib}
\addbibresource{bibliography.bib}
\usepackage{amsmath}
\usepackage{semantic}
\usepackage{mathrsfs}
\mathlig{-><-}{\rightarrow\leftarrow}
\newcommand{\sep}{\:|\:}
\newcommand{\oracle}{\ensuremath{\mathscr{I}}}
\newcommand{\staticlang}{\ensuremath{\mathscr{L}_s}}
\newcommand{\aware}{\ensuremath{A_a\varphi}}
\newcommand{\know}{\ensuremath{K_a\varphi}}
\newcommand{\anso}{\ensuremath{R_{\Phi}}}
\newcommand{\ansa}{\ensuremath{R_{a}}}
\newcommand{\powset}{\ensuremath{\mathscr{P}}}



\begin{document}
\onecolumn
\begin{center}
	\section*{Summary}\label{sec:summary}
\end{center}
This paper investigates the disparity between win-rates of opposing teams in 
social deduction games with an abundance of special roles. To do  this, an 
approach based on epistemic logic and common sense is used to create a 
simulation that emulates physical games of this type. The foundation of the 
game is based on other approaches within the same field, combining the ability 
of agents to inquire other agents for information, as well as giving them the 
ability to, naively, keep track of truths and lies. \\
The field of epistemology lays the groundwork of inference in the simulation, 
as well as providing a formal framework for the description of behaviour 
reliant on logical formulae within the simulation. \\
The simulation was built with a truth-table approach, resulting in a heavy 
computational burden, but with the benefit of being easy to understand 
intuitively. \\
The simulation was run with a multitude of different team compositions, which 
resulted in the revelation that the most important factor to consider when 
aiming for the creation of balanced social deduction games is the ratio of 
mafia members to town members. Secondly, it was found that killing-heavy mafia 
teams perform better than any other compositions. Lastly, excluding such teams 
showed 
that teams with the ability to acquire new information quickly performed better 
than those who could stop the town from performing their nightly actions. 
\twocolumn
\title{Evening the Odds:\\ The Impact of Town-Opposed Roles in Social Deduction
	Games\\ \small An experimental analysis of the impact of various
	town-opposed roles in social deduction games, using epistemology}
\author{
	Hugin Juliansson Zachariasen \\ Aalborg University \\ hjulia21@student.aau.dk
	\and
	Kristian Hadberg Mikkelsen \\ Aalborg University \\ khmi22@student.aau.dk
	\and
	Lucas Mørk Frendorf \\ Aalborg University \\ lfrend21@student.aau.dk
	\and
	Nicklas Peter Kvist Gislinge \\ Aalborg University \\ ngisli21@student.aau.dk
}
\maketitle
\begin{abstract}
	This paper seeks to investigate and propose solutions to win-rate disparities that arise in social deduction games with many powerful roles. We look at the roles that can be introduced to the smaller, informed team in order for them to resist the increased power and capabilities of the larger, uninformed team.\\
	We propose a formal combination of dynamic epistemic logic and epistemic inquiry, to form a dynamic epistemic inquiry language. We use this language to model the knowledge that our agents use as a foundation for their actions.\\
	To investigate this, we created our own social
	deduction game simulation based on the online game
	'Town of Salem'\cite{TownOfSalem}.\\
	Our experiments conclude that, for the smaller, informed team, obtaining more information and consistently removing players from the larger team, proved to be the most beneficial strategy.
\end{abstract}

\section{Introduction}
When trying to solve a problem, many solutions may present themselves. However,
most of these will come with unintended side effects. In the world of social
deduction games, one specific problem has manifested itself. As more diverse
roles have been introduced, in order to keep the games fresh, new, and
exciting, an imbalance has presented itself in the win-rate between the
opposing teams. \\ \\ Most social deduction games have the same foundation: The
players are divided into two teams, one larger with uninformed players, and one
smaller with informed players. Each of the teams wants to eliminate the other,
with differing methods of accomplishing this. Often, the uninformed team will
have a few special roles assigned to it, in order to acquire information, to be
able to root out the smaller team. The smaller team typically tries to stay
hidden, while eliminating the players in the bigger team.\\ \\ Historically,
the larger team has typically been burdened by the assignment of many
low-impact roles to its players. As games have evolved, players have grown
bored with this dynamic. That has led to the introduction of more roles, in
order to excite the players who were previously assigned the low-impact roles.
But as their roles are no longer low-impact, this has skewed the win-rate of
their team in their favour. This is problematic from a balance perspective, and
therefore this paper will look at how one can analyse and introduce roles to
the opposing team, in order to equalize this imbalance. The roles that will be
analysed consist mainly of two types: information- gatherers and deniers.
Through this analysis we will be able to draw a general conclusion on the topic
of whether it is more important to gather information for your own benefit, or
to deny information to opposing entities.

This will be analysed by utilizing a self-developed, console-based simulation
of a social deduction game, incorporating a numeric approach to
decision-making, with a foundation in epistemic logic. The game will be
modelled with a variety of roles and actions, as well as voting strategies and
communicative actions.

A sizeable amount of research has been put into the field of social deduction
games. However, much of that research has been focused on strategies relating
to when and how to utilize different communicative actions\cite{commitment}.
Some focus on ways for the town to detect the mafia
members\cite{werewolf_stealth}, or protect key town members from the
mafia\cite{werewolf_nash_equilibrium}. Others focus on the different
decision-making strategies that can be implemented in multiagent systems, which
these games can be modelled
as\cite{modelling_multi_agent_epistemic_systems}\cite{multi_agent_epistemic_planner_common_knowledge}\cite{probibalistic_multiagent_systems}.

However, one thing that is noticeably absent from this research is analyses
with the focus of mitigating the previously mentioned side effect, that arises
when playing against a team consisting of mostly high-impact roles.
Additionally, there is minimal research on the ability to inquire other
players, regarding their knowledge of the world. The outcome of such an
analysis might be a game setup where everyone involved can have a role with
special abilities. This may also mean that players will be more engaged and
active throughout all stages of the game. Lastly, it may also grant some
insight into whether it is more beneficial to gather information for oneself,
or to deny information to others.\\ \\

In the following section \ref{sec:RelatedWorks}, we will review papers and
studies that have significantly contributed to our area of study.

We then describe, how we intend to model our social deduction game in section
\ref{sec:Modelling}. It outlines the rules of the game and its operational
mechanics. Key roles will be explained in detail, as well as how agents
communicate with each other.

Continuing on to section \ref{sec:DynamicInquiryLanguage}, this section begins
with an introduction to the terminology and the inquiry language. We then
expand this to a static inquiry language with agents and knowledge. Finally, we
conclude the section by expanding it to a dynamic inquiry language with model
updates.

For the implementation section \ref{sec:implementation}, we provide a detailed
walkthrough a few of the processes utilized in our implementation. We start by
introducing the various data structures and then explain how each 'phase' of
the game is computed.

In the final sections of the paper \ref{sec:results}, \ref{sec:discussion}, and
\ref{sec:conclusion}, we present all our findings from our experiments.
\section{Related Works}
A paper authored by Eger and Martens\cite{commitment} analysed the effects of
commitment to a plan in the game One Night Ultimate Werewolf. They modeled this
using Dynamic Epistemic Logic(DEL), allowing each agent to be aware of multiple
different possible worlds with alternating sets of facts, and sorting them
based on the least likelihood of it being true. This is achieved by ‘marking’
worlds that contradict statements made by other agents.

When an agent needs to choose what to base their next action on, they determine
that based on the number of marks of other players, or randomly, if they have
no facts. They proclaim this gives the agents a certain degree of planning.
However, due to the increasing amount of worlds generated by new actions
performed, many of them equivalent, a notion of commitment is introduced. This
means that they will stick with a certain plan until another plan is valued
significantly higher. They compare this commitment to a baseline of random
actions by villagers, resulting in a high win percentage for werewolves,
proving their hypothesis. In a game of both werewolves and villagers with
weighted worlds, they found that simply valuing the plans equally, that is
choosing the new plan if it is strictly better than the previous, resulted in
the lowest win percentage for werewolves. \todo{Speculate how this relates to
    what we do}\\ \\ To formally support a model where players can ask each other
questions, and also gain additional knowledge throughout the game, two
epistemological concepts must be introduced. The Interrogative Model of Inquiry
(IMI) and Dynamic Epistemic Logic are both central paradigms of formal
epistemology that support these features, and where one lacks, the other
excels. While IMI provides knowledge, reasoning, and inference in a multi-turn
information-gathering process between an Inquirer and Nature, DEL models how
each agent updates their knowledge in a multi-agent setting.

In 2014 Y.Hamami\cite{delimi} published a formalization of how to combine
these, which combines the information-gathering process in an inquiry with
respect to multiagent dimensions. As opposed to standard IMI or DEL, the
proposed DEL\textsubscript{IMI} sees information as acquired by either common
knowledge or from other agents in an interrogation. \\ The inclusion of agents
also introduces the notion of lying, that is answering an interrogative
question with non-factual information, closely representing the nature of
social deduction games. In reality, both the intrigue and the complexity of
social deduction games often arise from the discussions involved between
players, which we see as a core part of this type of game. While other works
related to social deduction games have resorted mostly to rely on simple
one-way actions, the DEL\textsubscript{IMI} model allows agents to interrogate
others, perhaps challenge their claims, find inconsistencies, or use logical
inference to establish presuppositions.\\ \\ To formally describe the concepts
related to the modelling of this simulation, we draw upon the foundations of
dynamic epistemic logic as well as formalisations for describing multi-agent
systems. Such sources are related to dynamic epistemic inquiry\cite{delimi},
modelling knowledge- and
information\cite{modelling_multi_agent_epistemic_systems}, planning with common
knowledge\cite{multi_agent_epistemic_planner_common_knowledge}, and
probabilistic approaches\cite{probibalistic_multiagent_systems}. All of these
were significant in the pursuit of knowledge regarding this area of research,
and in building an adequate vocabulary to describe actions, beliefs, knowledge,
and epistemology in general.
\section{Modeling}
As mentioned earlier, we wish to model a framework of games instead of one
specific one, in order to encapsulate as many parameters as possible. The game
that will be modeled will have two factions, henceforth known as the town and
mafia. The town wins when they have eliminated all mafia members, and the mafia
wins when their factions have the non-strict numerical majority of players. The
game is played out in rounds consisting of four phases, which happen in order,
until either win-condition is met: The night phase is where different roles
perform private or public actions to gain information or spread misinformation.
The morning phase is where public results of the night phase are revealed to
all players. The role of players who died during the night phase is revealed at
this time. The day phase is where players try to influence each other's
worldviews through communicative actions. The voting phase is where players
vote for who to lynch according to their individual worldviews. The voting
phase can end in a majority vote which will result in a lynching of that
player, killing them and revealing their role publicly, or in a tie, which will
result in the phase ending with no lynching. The possible roles that a player
can play in this modeling of the game, and their associated actions, are: Town
Roles: A Villager has no special actions. The Sheriff must choose one player
each night and privately gain information regarding their faction. The investigator must choose one
player each night and privately gain information regarding a selection of roles that that player
may have, one of them being true. The Doctor must choose one player each night
and protect them from any harm that may come to them during the night. The
Escort must choose one player each night and prevent them from performing any
actions during the night phase. The Veteran may choose to go on alert during
the night phase, if they do so, then anyone visiting them at night will die.
The Vigilante may choose to kill any one player during the night phase, if that
player is revealed to be a member of the town during the morning phase, the
vigilante loses their killing ability, and will die the following morning.
Mafia Roles: The Godfather must choose one player each night who will die. The
Mafioso performs the killing on the Godfathers behalf. The Framer must choose
one player each night to frame. A framed player appears to be a member of the
mafia if chosen by the Sheriff on this night. The Blackmailer must choose one
player each night to blackmail. A blackmailed player can only choose the “do
not say anything” communicative action.

Now, to simulate the games as closely as possible to the real world we use a 
foundation built upon a probabilistic approach to decision making, and an 
epistemological approach to the acquisition and refreshing of 
knowledge\cite{commitment}.
Their core principles are that players keep track of all possible worlds that
may be possible based on their own knowledge, and the public knowledge
available to them. They keep track of beliefs and lies by “marking” worlds
which contradict statements made by other players. Due to the fact that there
should always be more players belonging to the town than the mafia, and that
the players of the town have no reason to lie, it is presumed that the world
with fewest “marks” must be the true world, which the player will base their
decisions on. A few modifications to their methodology have been made, in order
to better suit a multi-round game. First, the communicative actions for the
game have been altered slightly. In this simulation, players can choose between
6 different communicative actions:

\begin{enumerate}
    \item Claim to have a certain role.
    \item Claim that someone else has a certain role.
    \item Claim to have performed a nightly actions, with some result.
    \item Inquire another player regarding their role.
    \item Inquire another player regarding their beliefs about other players.
    \item Do not say anything.
\end{enumerate}
The purpose of these actions are to mimic the in-person game as closely as
possible. Players will each be able to take 3 communicative actions before the
phase concludes. Players choose which actions to take, based on which action
will influence others’ most towards their own world view. If multiple actions
are in accordance with their world view, then they may choose to inquire other
players in order to maximize their information gain. When public information is
revealed, like the role of a killed player, all players must update their
worldviews, eliminating all worlds not in accordance with this information.
This will slowly limit the possible worlds, granting more information to the
town. The marks generated by communicative actions are associated with the
player that generated it, meaning that whenever public information is revealed
about that player, villagers may either discard or reinforce said marks,
depending on whether the player was revealed to be a member of the mafia,
meaning that may have had incentive to lie, or if they were a villager, meaning
they should have only told the truth. Another thing, kept track of by the
players, are the active and inactive worlds. An active world is a world that is
true based on the public and private information available to a player. An
inactive world is a world that is true based on the public information
available, but false based on the private information available. The need for
inactive worlds, is that a player cannot assume that other players are aware of
private information, so these worlds must be included when deciding which
communicative actions to perform during the day phase. An example: Player 1 is
a seer, and has used their nightly action to look at the faction of player 2,
which gave them the private information that player 2's faction is of the town.
Player 1 can now mark all worlds where player 2 is part of the mafia as
inactive, as they are factually false, but since this is not public information
player 1 knows that not everyone knows this, and these worlds must therefore
still be considered when choosing communicative actions, even if just for
knowing which worlds to avoid promoting.

\\\\

Social deduction scenarios can be expressed or modeled differently depending on the methods used to solve them. While other noteworthy fields such as machine learning could possibly also be used, we see an intriguing similarity between epistemology and social deduction, which lies in the nature how these games played. Often it's a game between players (agents), which need to keep track of who said what, and who knows what, and what these truths entails. Additionally, recent developments in the field have largely been focused on exactly these multi-agent systems and artificial intelligence. 

\section{Dynamic Inquiry Language}
The terminology and its notation used throughout this article will closely
align with Yacien Hamami \cite{delimi}, although not all aspects will be
presented here. Hamami uses an awareness operator to distinguish between explicit/implicit knowledge and an oracle to avoid logical omniscience, we however disregard this, as it is not within the scope of this article. 

The motivation for this section is to formally describe our
modeling prerequisites, and to argue for the correctness of these. This section will cover what is necessary to reach
our dynamic inquiry epistemic logic model. First we define a simple inquiry language, then extend this to our static inquiry language by introducing agents and knowledge, finally culminating in defining our dynamic inquiry epistemic language with information and model updates. An overview of notations is provided in \cref{notationalschema}, for easy references to the used notation. 

\newpage  

\begin{table}[t]
	\caption{Notional Schema \label{notationalschema}}
	\begin{tabularx}{\linewidth}{p{0.40\linewidth}X}
		\toprule
		
		\multicolumn{2}{l}{{\underline{Symbols:}}}                                       \\
		\textbf{P} & a countable set of atomic propositions \\
		\textbf{Ag} & a countable set of agents \\
		\textit{M} & an IMI epistemic model \\
		\textit{V} & atomic valuation function \\
		$\sim_a$ & binary equivalence function for a given agent \textit{a} \\
		\textbf{\powset} & the power set \\
		\oracle & inquiry language \\
		\staticlang & static epistemic language \\
		\dynlang & dynamic epistemic language \\
		
		\multicolumn{2}{l}{{\underline{Operators:}}} \\  
		$\gamma$ & proposition in the inquiry language \\
		$\varphi$ alt. $\psi$ & proposition in the static inquiry language \\
		$K_a\varphi$ & knowledge operator \\
		$R_a((\varphi_1,...,\varphi_n), \varphi_i)$ & agent answer operator \\
		\pubop & public announcement operator \\
		\agquestop & agent question operator \\
		\infop & inference operator \\
		
		\bottomrule
	\end{tabularx}
\end{table}

Firstly, we will define our \textit{inquiry language}. Recall that in \textbf{DEL}s,
factual information is represented in the propositional language and is
therefore also the inquiry language, denoted by \oracle, defined in BNF format: 

\begin{align}
	\gamma::= p \sep\:\neg\gamma\sep(\gamma\land\gamma) \label{eq:1}
\end{align}

where \textbf{P} is a countable set of atomic propositions, $p \in\mathbf{P}$ and $\gamma$ reads as a proposition is either a fact, the negation of a proposition or the conjunction of propositions. The \textit{inquiry language} $\oracle$ is simply the symbol representing factual information based on Hintikka \cite{hintikka88}. It should be perceived as being the source of common knowledge between agents, and therefore also do not have any notion of which agents know what, or any distinguish-ability between worlds.

We can now extend the previous definitions to describe the \textit{static IMI
	epistemic language} \staticlang\: as follows:
\begin{align}
	\begin{split}
		\varphi ::= p \sep\neg\varphi\sep(\varphi \land \varphi) \sep K_a\varphi \sep R_a((\varphi_1,...,\varphi_n), \varphi_i) \\ \text{where p} \in \text{\textbf{P}}\text{, a} \in \text{\textbf{Ag}}, n \in \mathbb{N}, i \in 1..n \label{eq:2}
	\end{split}
\end{align}
where \textbf{Ag} is a set of agents, the \textbf{knowledge operator} $K_a\varphi$ reads "agent \textit{a} implicitly knows that $\varphi$",  $R_a((\varphi_1,...,\varphi_n,), \varphi_i)$ reads as "$\varphi_i$ is the answer agent \textit{a} will provide to question $\varphi_1,...,\varphi_n$". \\

We can now define the \textit{IMI epistemic model}, a tuple:
\begin{align}
	M = \langle W, \sim_{a\in Ag}, V, R_{a\in Ag}\rangle \label{eq:3}
\end{align}
where:
\begin{itemize}
	\setlength\itemsep{-0.4em}
	\item W is non empty set of worlds.
	\item $\sim_{a\in Ag} \subseteq W \times W$ is a binary equivalence relation representing the indistinguishability relation of agent $a$, meaning that the relation is reflexive, transitive and symmetric.
	\item $V : W \rightarrow \mathscr{P}(\mathbf{P})$ is the atomic valuation function, which yields the propositions which are true in each world. 
	\item $R_a : W \rightarrow \powset(\staticlang) \times \staticlang$ is the answering rule of agent \textit{a} associating each world $w \in W$ with a pair of the form \aset $\:$ where $(\varphi_1,...,\varphi_n) \subseteq \staticlang$ and $\varphi_j \in \staticlang$.
\end{itemize}
Where $\mathscr{P}$ denotes the power-set. We introduce the answer functions, so agents are able to model or predict in possible worlds, based on what they know, what another agent will answer. They are modeled as pairs of a series of questions to one answer, because the provided answer to a series of questions can be a conjunction of propositions. We hypothesize that answering correct accordingly to a world, should imply some sense of truthfulness to this world. By our static epistemic language \staticlang\: and model $M$, we can define the following semantics:

\begin{gather}
	M, w |= p \iff p \in V(w) \label{sem:1}\\
	M, w |= \neg\varphi \iff M, w \not\models \varphi \label{sem:2}\\
	M, w |= (\varphi \land \psi) \iff M, w |= \varphi, M, w |= \psi \label{sem:3}\\
	M, w |= \know \iff \forall u\in W, u \sim_{a} w \implies M, u |= \varphi \label{sem:4}\\
	M, w |= R_a \iff \aset \in R_a(w) \label{sem:5} 
\end{gather}
While \cref{sem:1}, \cref{sem:2} and \cref{sem:3} are trivial, \cref{sem:4} explains that the knowledge of some proposition \proposition is satisfied in world $w \in W$ if and only if it is also satisfied in all epistemically equivalent worlds by the indistinguishability relation $\sim_a$ for agent \textit{a}. That is, the agent does not distinguish between worlds, which by the knowledge in these do not have conflicting propositions. The answer function \cref{sem:5} is likewise only satisfied in a world given a model, when some pair of questions and answer is in set of the agent answering operator. 

\subsubsection*{Information Update}
To extend our static language \staticlang\: to a dynamic language, which we will donate as \dynlang, we introduce three new operators, the \textit{public information update operator}, the \textit{agent question operator} and the \textit{inference operator}: 

\begin{equation}
	\pubop \sep \agquestop \sep \infop \label{eq:6}
\end{equation}
The information update operator \pubop \: reads as "after public announcement of $\psi$, then \proposition is the case", with $\varphi, \psi \in \staticlang$. Formulas of the form \agquestop\: are read as "\proposition is the case after agent \textit{a} has asked the question $(\varphi_1,...,\varphi_n)$ to agent \textit{b}", and \infop\: is read as "\proposition is the case after agent \textit{a} has logically inferred $\psi_c$ from premises $\{\psi_1,...,\psi_m\}$". All these require some notion of model update, which we will define by a given model similar to \cref{eq:3}, and a model $M'$ as:
\begin{flalign}
	M|\varphi &= M' \label{eq:4} \\
	M' &= \langle W', \sim'_{a\in Ag}, V', R'_{a\in Ag}\rangle \label{eq:5}
\end{flalign}
\\ 
In which $\proposition \in \dynlang$. Note that the expression $M|\varphi$ in \cref{eq:4} refers to a model update. This should be understood as $M'$ is the model in which all worlds where $\varphi$ is false or are not contained, are removed. The members of $M'$ is then given by:

\begin{itemize}
	\item $W' := \left\{ w' \in W \land M, w' \models \proposition \right\}$ 
	\item $\sim'_a := \sim_a \cap \:(W' \times W')$, for all $a \in Ag$
	\item $V' := V | W'$
	\item $R'_{\Phi, a\in Ag} := R_{\Phi, a\in Ag} | W'$ 
\end{itemize}
While all three operators are inherently information updates and follow the above listed transformations, the \textit{agent question operator}, which we will denote as $Q_{A}?$ with $Q_A = (\varphi_1,...,\varphi_n)$ and the \textit{inference operator} denoted by $I$, where $I = \{\psi_1,...,\psi_m \}\hookrightarrow \psi_c$, have additional constraints. An agent should not be able to ask a question without knowledge of it's presupposition and it should be a part of the opposing agent's answer set. This can be described as:
\begin{itemize}
	\item When $Q_A = (\varphi_1,...,\varphi_n) \in \powset(\staticlang)$, then if there exists $\varphi \in Q_A$ such that $M, w \models K_b\varphi$ and $(Q_A, \varphi_i) \in R_b (w)$, with $a, b \in Ag$ and then:
	\begin{align}
		M^{a,b}_{Q_A}?(w) := M |\varphi_i
	\end{align}
	\item Otherwise, $M^{a,b}_{Q_{A}}?(w) := M$
\end{itemize}
Such that the updated model satisfy the worlds in which \proposition\: is also satisfied. Informally this can described as the updated model $M|\varphi_i$ contains the worlds in which the answer to $Q_A$ is also the expected answer, based on the answering rule \cref{sem:5}. Additionally we constrain the operator, such that agent $a$ asking question $Q_A$ should have knowledge about at least one of the propositions. Since questions are regarded as potentially a series of questions, and during an interrogation an agent can infer from intermediate answers, this is necessary. We formally describe the precondition to $Q_A?$ in \dynlang, where $Q_A = (\varphi_1,...,\varphi_n)$:

\begin{gather}
	pre_{a,b}(Q_A) := K_a\Biggl(\bigvee\limits_{i\in 1,n}\varphi_i\Biggr)
\end{gather}
The inference operator $I$ as we described earlier, can simply denoted as a model update containing the conclusion based on the premises ${\psi_1,...,\psi_m} \hookrightarrow \psi_c$. We formally describe this as:

\begin{itemize}
	\item Let $M$ be a $DEL_{IMI}$ model, $\psi_1,...,\psi_m, \psi_c \in \staticlang$, $a\in Ag$, $I=\{\psi_1,...,\psi_m\} \hookrightarrow \psi_c$, then the model update is given by: 
	\begin{gather}
		M^a_I(w) := M|\psi_c
	\end{gather}
\end{itemize} 
Similar to the definition of the precondition to the \textit{agent operator}, we also require the agent to be knowledgeable about both the premise, and that the conclusion follows from these. We describe this by:
\begin{gather}
	\nonumber pre_{a,b}(I) := \\ \bigwedge\limits_{i\in1,m}K_a\psi_i \land K_a\Biggl(\Biggl(\:\bigwedge\limits_{i\in 1,m}\psi_i\Biggr) \hookrightarrow \psi_c \Biggr)
\end{gather}
We now have the prerequisites for defining the semantics of our dynamic language \dynlang. They are based on the previous listed in \crefrange{sem:1}{sem:5} describing the \staticlang, extending it with the newly introduced \textit{information update operator}, \textit{agent question operator} and \textit{inference operator}:
\begin{gather}
	M, w \models [\psi!]\varphi \iff M, w \models \psi \implies M|\psi, w \models \varphi \\
	M, w \models [Q_A?]_{a,b}\varphi \iff M, w \models pre_{a,b}(Q_A) \implies M^{a,b}_{Q_A?}(w), w \models \varphi \\
	M, w \models [I]_a\varphi \iff M, w \models pre_a(i) \implies M^a_I(w), w \models \varphi
\end{gather}
Informally, these simply explain that the semantics for the update of worlds only hold if their preconditions hold, which then in turn implies that the worlds accessible after the model update includes the conclusive proposition. \\

We have now formally described the definitive language of \dynlang, which lays the foundation for our  implementation of simulating our modeled social deduction game. 


\subsection{Example}
A round of our modeled game can be expressed in the dynamic inquiry language. In a model defined as \cref{eq:3}, we let agents \textbf{Ag} = \{a, b, c\}, where \textit{a} is a sheriff and $\varphi$ = "agent \textit{a} is a sheriff", and agent \textit{b} and $\psi$ = $(\psi_1 \lor \psi_2 \lor .. \lor \psi_{n-1})$ = "agent \textit{b} is a villager or godfather or .." for all \textit{n} roles, except sheriff, which agent \textit{a} knows himself is. Agent \textit{c} will be godfather represented by $\gamma = (\gamma_1 \lor \gamma_2 \lor .. \lor \gamma_{n-1}) $. In our simulation, we generate $W$ by creating all possible worlds, based on the aforementioned propositions, by appendix C this is simply the function call \lstinline{generateWorlds()}. The round then starts with communicative actions, where agent \textit{b} says "I am villager", represented by $(p \lor \neg p)$, and by inference if what agent \textit{b} says is correct, then it is also true that they are their mentioned role. This is represented by $p \rightarrow \psi_1$ where $\psi_1$ is agent \textit{b} is a villager. Agent \textit{c} also says they are a villager, represented by $q$. 

\begin{lstlisting}[basicstyle=\footnotesize\ttfamily, numbers=left, xleftmargin=0.5cm, firstnumber=17, caption={Snippet from appendix C}, captionpos=b]
func communicate():
	world = getMostPlausibleWorld()
	action = self.getAction(world)
	
	if (action is inqure or accuse)
		player = getHighestInformationGainPlayer()
		question = getHighestInformationQuestion(player)
		communicativeAction = self.ask(player, question)
	...
	updatePossibleWorlds(communicativeAction)
\end{lstlisting}
When the agents communicate this, they choose which action to take, based on what communicative action will result in the most amount of worlds being marked as inactive. While $p \lor \neg p$ doesn't directly cause worlds to be, the result whenever we find out whether $p$ or $\neg p$ will. This also includes following implications from the resulting propositions. Afterwards, we update our worlds, to include the newly acquired information. 

This whole sequence can be expressed by the following formula:

\begin{align}
	\begin{split}
		G = K_a(\varphi \land \psi \land\gamma)\land p \rightarrow \psi_1 \land q \rightarrow \gamma_1 \\ \land (p \lor \neg p) \land (q \lor \neg q) \label{eq:7}
	\end{split}
\end{align}

When the night phase comes, agent \textit{a} will interrogate agent \textit{b}, resulting in now agent \textit{a} knowing the role of agent \textit{b}. During the same night, agent \textit{b} is killed by the mafia, and then publicly revealed. Recall from \cref{eq:6}, we can say $\varphi_1!G\land\varphi_1$, which informally is that everything before the public announcement and the announcement itself is now applicable. In our simulation, this is done by calling \lstinline[]{updatePossibleWorlds(information)}, displayed in \ref{lstlisting:2} 
\begin{lstlisting}[basicstyle=\footnotesize\ttfamily, numbers=left, xleftmargin=0.5cm, firstnumber=31, caption={Snippet from appendix C \label{lstlisting:2}}, captionpos=b]
func updatePossibleWorlds(information)
	foreach (player in game.Players)
		foreach (possibleWorld in player.possibleWorlds)
			possibleWorld.UpdateMarks(information)
\end{lstlisting}
Which simply iterates over all players and all their respective worlds, and updates the information. Now recall back to \cref{eq:7}, after the newly acquired information of $\varphi_1$ we can simplifying the expression, which results in:
\begin{align}
G' = K_a(\varphi \land \psi_1 \land \gamma_2) \land p \land \neg q
\end{align}
Agent \textit{a} now knows that agent \textit{c} is the godfather.
\section{Implementation}\label{sec:implementation}
With the theory explained, we can now explore our implementation, which aims to
be as true to our formal description as possible. We will start by laying the
foundation of the most pivotal data structures, followed by a walk-through of
the implementation of each phase of the game.

\subsection{Foundational Data Structures}\label{sec:foundationalDataStructures}
To provide some context for the upcoming sections, we will first describe the
data structures they are based upon.

A game consists of some $n$ number of agents, wherein each agent has $m$ amount
of worlds. $m$ is defined as described in \cref{sec:TheKnowledgeOfAgents}. A
world, often referred to as a possible world, describes a possible combination
of roles and agents. Each world has a list of roles, which represents that
specific world's role composition. This relation is depicted in
\cref{fig:DataStructuresUML}. \renewcommand{\umlfillcolor}{white}
\renewcommand{\umldrawcolor}{blue}
\begin{figure}[H]
	\centering
	\scalebox{0.8}{
		\begin{tikzpicture}
			\begin{class}[text width= 3cm]{Game}{-0.5, -0.63}
				\attribute{agents : list}
				\operation{runDayPhase()}
				\operation{runNightPhase()}
				\operation{checkFinished()}
			\end{class}
			\begin{class}[text width=4.5cm]{Agent}{5,0}
				\attribute{worlds : list}
				\attribute{communicationLog : list}
				\attribute{role : string}
				\attribute{isAlive : boolean}
				\operation{updateWorlds(information)}
				\operation{nightAction()}
				\operation{communication()}
			\end{class}
			\begin{class}[text width=4cm]{World}{-0.5,-4.85}
				\attribute{possibleAgents: list}
				\attribute{marks : integer}
				\attribute{isActive : boolean}
				\attribute{isPrivateActive : boolean}
			\end{class}
			\begin{class}[text width=3.5cm]{PossibleAgent}{5,-5}
				\attribute{actualAgent: agent}
				\attribute{isAlive : boolean}
				\attribute{possibleRole : string}
			\end{class}
			\unidirectionalAssociation{Game}{}{1..n}{Agent}{}{}
			\unidirectionalAssociation{Agent}{}{1..m}{World}{}{}
			\unidirectionalAssociation{World}{1..n}{}{PossibleAgent}{}{}
			\unidirectionalAssociation{PossibleAgent}{*..1}{}{Agent}{}{}
		\end{tikzpicture}}
	\vspace*{3mm}
	\caption{Simplified UML diagram of data structures, where n is the amount of agents and m is the amount of worlds for each agent.}
	\label{fig:DataStructuresUML}
\end{figure}
We argue this encapsulates the model both described in \cref{sec:Modelling}, and the formalisation in
\cref{sec:DynamicInquiryLanguage}, equation \cref{eq:3}, most noteworthy is the relation between $W$ and \textbf{Ag}. The individual elements of \cref{fig:DataStructuresUML} will be explained further in the following sections.
\subsubsection{The Knowledge of Agents}\label{sec:TheKnowledgeOfAgents}
The knowledge and logical propositions of agents has been chosen to be
represented with a truth-table approach, which refers to the possible worlds
which store all possible pairings of agents and roles. The following equation
shows the size of the list given agent count $A$ and the number of agents with
role $R$, this expression also assumes that agents that are not assigned a
role, will become villagers.
\begin{equation}
	W: A,R \in  \mathbb{N} \mapsto \frac{A!}{R!(A-R)!}\
\end{equation}

For each subsequent role, $R_1, R_2, .., R_n$ that is added, the complexity
expands with:
\begin{equation}
	\begin{gathered}
		W: A, R_0 = 0, R_1, ..., R_n, n\in  \mathbb{N} \mapsto \\
		\frac{\prod\limits_{i=0}^{n}(A-\sum\limits_{j=0}^{i}R_j)!}{\prod\limits_{i=1}^{n}R_i!(A-\sum\limits_{j=1}^{i}R_j)!}
		\label{eq:numofworlds}
	\end{gathered}\footnote{If the number of agents equals the
		number of roles, then the formula collapses into A!.}
\end{equation}

Each agent excludes possible-worlds that are not in accordance with the role
that they have been assigned.\\ \textit{For example}: Agent \textit{a}, if
given the role of sheriff, will exclude all possible-worlds where their role is
not sheriff, while another agent \textit{b} may keep some of those
possible-worlds, since they do not know the role of \textit{a}.\\ In our
implementation, this means that for agent \textit{a} the worlds, where they are
not a sheriff are privately inactive. Furthermore, each agent also contains a
list of who have made what claims, and with this, they can extrapolate whether
that claim is true or false in a given possible-world. This is represented by a
list of communicative actions, and the amount of lies of agents, is stored in
the possible worlds as \textit{marks}. If a world has many marks, it implies
that many agents have lied if this is the true world, which would be unlikely.
It is also important to remember, that we keep track of who contributed to each
mark in a given world. This will be explained further in the following
sections.

\subsection{Updating Knowledge}\label{sec:UpdatingKnowledge}
When a new fact is revealed, such as when an agent dies and their role is
revealed, all agents will immediately exclude all possible-worlds in which the
dead agent is anything other than their revealed role. Implementation wise,
this means that the worlds which are not in accordance with the new information
are set as inactive\footnote{Practically this is equivalent to them being
	deleted}. This corresponds to the model update defined in \cref{eq:5}. An agent
dying is the way that most roles learn new facts. \\ Depending on the role that
died, all agents update the marks that the dead agent contributed to.\\
\textit{For example}: A sheriff has a nightly action which allows them to gain
more information than regular villagers. Therefore, the marks that a sheriff
has contributed to, should be weighed more heavily than those of other roles.
To emulate this, other agents add an additional mark to any marks that were
contributed to by the sheriff, to further denote the importance of their
opinions. \\ Conversely, should the dead agent be revealed to be a member of
the mafia, all agents will detract a mark from the marks that the dead agent
contributed to. Due to mafia member have incentive to deceive the other agents.

\subsection{Beginning of the Game}\label{sec:beginningOfTheGame}
With the foundational data structures explained, we can begin the game, which
is done in accordance with the pseudocode in \cref{alg:StartGame}.
\begin{algorithm}[H]
	\caption{StartGame}
	\begin{algorithmic}[1]
		\Function{StartGame}{}
		\State loadGameConfiguration()
		\State generateWorlds()

		\While{!gameFinished}
		\State dayPhase()
		\State nightPhase()
		\State checkFinished()
		\EndWhile
		\EndFunction
	\end{algorithmic}\label{alg:StartGame}
\end{algorithm}
\setcounter{algorithmcaption}{0}
\captionof{algorithmcaption}{Pseudocode for starting and running the
	game.}
The game begins by loading the role-configuration for that specific game, and
then generating the worlds for all agents. It then goes through the day- and
night phases, followed by checking if either team has won.

\subsection{Day Phase}\label{sec:dayPhase}
The day phase is split into two sub-phases: communication and voting. These
phases are performed in order, and controlled by \cref{alg:dayPhase}.
\begin{algorithm}[H]
	\caption{Day phase}
	\begin{algorithmic}[1]
		\Function{dayPhase}{}
		\ForAll{a $in$ game.Agents}
		\State a.communicate()
		\EndFor
		\ForAll{a $in$ game.Agents}
		\State votes.append(voting(a))
		\EndFor

		\State killMostVoted(votes)
		\EndFunction
	\end{algorithmic}\label{alg:dayPhase}
\end{algorithm}
\setcounter{algorithmcaption}{1}
\captionof{algorithmcaption}{The order of sub-phases within the day phase.}
First all agents communicate, then they all vote. The voting function, returns the target agent, which the calling agent, believes is the most suspicious and adds it to the \textit{votes} list.

\subsubsection{Communication}\label{sec:communication}
The communication sub-phase is where agents share information with each other
based on the marks in their active worlds. Communicative actions are the main
way to contribute to marks in a given possible-world. The way that a given
agent chooses who to target with their communicative action is chosen in
accordance with \cref{alg:GetHighInfoGainP}.
\begin{algorithm}[H]
	\caption{GetHighInfoGainA(me, tWorlds)}
	\begin{algorithmic}[1]
		\If{nonAccused.Count $>$ 0 OR Game.Round = 0}
		\State $(a, wID)$ $\gets$ GetNonAccused(me)
		\If{wID != -1}
		\State \Return (a, wID)
		\EndIf
		\EndIf

		\State (a, wID) $\gets$ GetMostContradictory(me, tWorlds)
		\If{wID != -1}
		\State \Return (a, wID)
		\EndIf

		\State w $\gets$ random world from me.Worlds \State
		a $\gets$ random agent from w.PossAgents \State \Return (a, index of w in
		me.Worlds)
	\end{algorithmic}\label{alg:GetHighInfoGainP}
\end{algorithm}
\setcounter{algorithmcaption}{2}
\captionof{algorithmcaption}{The base algorithms that handles the communicative logic.}
Where the parameter \textit{me} refers to the agent deciding an action, and \textit{tWorlds} is the top 1\% least marked worlds for \textit{me}. To get the highest information gain, \cref{alg:GetHighInfoGainP} returns an
agent based on different factors.
First, it prioritizes non-accused agents as shown in
\cref{alg:GetNonAccused}.
Then finds the most contradictory information as shown in
\cref{alg:GetMostContradictory}. We do this, to first gain some information on every player, and then target the one with most contradictions. \\
\begin{algorithm}[H]
	\caption{GetNonAccused(me)}
	\begin{algorithmic}[1]
		\If{nonAccused.Count = 0 AND Game.Round = 0}
		\State nonAccused $\gets$ all agents in me.Worlds[0]
		\EndIf

		\ForAll{communication in me.CommunicationLog}
		\State nonAccused.Remove(communication.target)
		\EndFor

		\State nonAccused.Remove(me)

		\If{nonAccused.Count != 0}
		\State \Return nonAccused[random]
		\EndIf

		\State \Return (null, -1)
	\end{algorithmic}\label{alg:GetNonAccused}
\end{algorithm}
\setcounter{algorithmcaption}{3}
\captionof{algorithmcaption}{Returns an agent that haven't been accused yet. nonAccused is a local list of agents which have not yet been targeted.}
Each agent initialises a list of all other agents at the
start of the first communication phase. Whenever an agent targets another agent
with a communicative action, they remove the target agent from their individual
list. The target will also be removed from the other agents lists, before the target for their communicative action is decided. When all agents have been targeted with one communicative action, some knowledge on all agents is available.They will use to determine
the new best target for their communicative
actions. This is done in the next step of \cref{alg:GetHighInfoGainP},
by invoking \cref{alg:GetMostContradictory}.
\begin{algorithm}[H]
	\caption{GetMostContradictory(me, tWorlds)}
	\begin{algorithmic}[1]
		\State roleCounts $\gets$ getRoleCounts(tWorlds)

		\ForAll{w in tWorlds}
		\ForAll{a in w.PossAgents where a != me AND a is alive}
		\State highest $\gets$ roleCounts[a].Max()
		\State secHighest $\gets$ roleCounts[a].SecondMax()

		\State contraScore $\gets$ secHighest / highest
		\If{contraScore $>$ maxScore}
		\State mostContradictory $\gets$ a
		\State maxScore $\gets$ contraScore
		\State wID $\gets$ index of w
		\EndIf
		\EndFor
		\EndFor
		\State \Return (mostContradictory, wID)
	\end{algorithmic}\label{alg:GetMostContradictory}
\end{algorithm}
\setcounter{algorithmcaption}{4}
\captionof{algorithmcaption}{Returns an agent with the most conflicting role claims. \textit{roleCounts} is the amount of different roles, an agent is in tWorlds}
This algorithm outputs the agent which is the most suspicious due to being
suspected of being the highest difference of roles in the top 1\% of active possible-worlds.\\
This whole process is described in the following high-level algorithm
\ref{alg:communicate} for how one agent handles their turn in the communication
sub-phase.
\begin{algorithm}[H]
	\caption{Communication}
	\begin{algorithmic}[1]
		\Function{communicate}{}
		\State world $\gets$ getMostPlausibleWorld()
		\State action $\gets$ self.getAction(world)

		\If{action is inquire or accuse}
		\State a $\gets$ getHighestInfoGainA(me, tWorlds)
		\State question $\gets$ getHighestInfoQuestion(a)
		\State commAction $\gets$ me.ask(a, question)
		\ElsIf{action is defend}
		\State commAction $\gets$ me.claimRole()
		\EndIf

		\State updatePossibleWorlds(commAction)
		\EndFunction
	\end{algorithmic}\label{alg:communicate}
\end{algorithm}
\setcounter{algorithmcaption}{5}
\captionof{algorithmcaption}{High-level pseudocode of an agents turn during the
	communication sub-phase.}
It describes how an agent's turn consists of finding their most plausible world,
deciding what actions to take, choosing their target, as described above, and
then updating their worlds' marks according to the outcome of the action.
The updating of their worlds is performed in accordance with
\cref{alg:updatePossibleWorlds}, seen below.
\begin{algorithm}[H]
	\caption{Update possible worlds}
	\begin{algorithmic}[1]
		\Function{updatePossWorlds}{information}
		\ForAll{a $in$ game.Agents}
		\ForAll{world $in$ a.Worlds}
		\State world.UpdateMarks(information)
		\EndFor
		\EndFor
		\EndFunction
	\end{algorithmic}\label{alg:updatePossibleWorlds}
\end{algorithm}
\setcounter{algorithmcaption}{6}
\captionof{algorithmcaption}{The updating of possible worlds. \textit{information} refers to either a communicative action or when a player dies.}
Whenever an agent gets an answer for their communicative action, or when other information is gained, all agents update their
world view by invoking \cref{alg:updatePossibleWorlds}. The \textit{UpdateMarks} function reviews the new information, and checking if the information is true or false in that world. If the information is false then a mark is added otherwise nothing happens.\\

The phase ends when a variable amount of communicative actions have been
performed by each agent, which, in this implementation, has been arbitrarily
set to 3.

\subsubsection{Voting}\label{sec:voting}
When communication and the updating of possible-worlds has finished, the game
moves on to the voting phase. Each agent chooses who to vote for, in a very
similar manner to the choice of target for their communicative action
previously shown in \cref{alg:GetHighInfoGainP}. They chose the active world
with the least marks, i.e. the one they find to be the most likely. If there
are multiple, then they pick randomly between them. They then select a random
member of the mafia from that world, and vote for that agent.

\subsection{Night Phase}\label{sec:nightPhase}
The night phase is also broken down into two sub-phases: the night phase and
the nightly actions phase.

\subsubsection{Nightly
	Actions}\label{sec:nightPhaseNightlyActions}
Nightly actions are performed during the night phase, and are tailored towards
each role. For each role, we define their individual most likely world $W$, as
the world with the least amount of marks for them. \\ An example of how a
nightly action is performed can be seen in \cref{alg:nightPhase}.
\begin{algorithm}[H]
	\caption{Night action}
	\begin{algorithmic}[1]
		\Function{performNightAction}{}
		\If{me.role == Sheriff}
		\State world $\gets$ getMostPlausibleWorld()

		\State susAgents $\gets$ getSortedPossMafAgents(world)
		\State susAgents.remove(me.previousTargets)

		\State me.actResult $\gets$ interrogate(susAgents[0])
		\EndIf
		\EndFunction

		\Statex
	\end{algorithmic}\label{alg:nightPhase}
\end{algorithm}
\setcounter{algorithmcaption}{7}
\captionof{algorithmcaption}{An example of a nightly action performed by the
	sheriff. \textit{interrogate} sets all worlds that are not in accordance with the new information to privately-inactive}
The sheriff finds who the mafia members are in $W$, and then targets one of
them, which he has not targeted before. Other agents perform their actions
similarly, such as the godfather trying to kill whomever is the sheriff in $W$.
The doctor tries to save whomever is the sheriff in $W$, etc.

\subsubsection{Morning Phase}\label{sec:morningPhase}
The morning phase begins after the night phase. Its only purpose is to publicly
announce the public effects of the nightly actions, and then make all agent
update their world-views, by using the previously discussed
\cref{alg:updatePossibleWorlds}. However, in the implementation facts are
acquired as soon as an action has occurred. This change does not affect the
game, as all agents already have declared which night action they want to
take.\\ As facts are revealed, possible-worlds are de-activated, meaning that
one may slowly infer the roles of other agents. Furthermore, as people die,
marks are added or altered based on how trustworthy the given victim's role
was. \\ It is important to note, that while the semantics include inference in
the form of propositions, and communicative actions can be expressed in the
propositional language, this is not included in our implementation. However,
since we include marks, and our decisions are based on these, we denote this as
weak inference of the communicative actions.

\subsection{Detailed run through}\label{sec:ARoundOfTheGame}
A round of our modelled game can be expressed in the dynamic inquiry language.
In a model defined as \cref{eq:3}, we let there be 10 agents \textbf{Ag} = \{a,
b, c, ..., j\}, where \textit{a} is a sheriff, \textit{b} is a godfather, and
the rest are villagers. This round is viewed from the sheriffs perspective, so
he does not know the roles of the other agents, only the amount of each role
that is in the game, i.e. 1 sheriff, 1 godfather, and 8 villagers. He then has
a set of propositions \{$\varphi$, $\psi$\, $\gamma$, ...\} where $\varphi$ =
"agent \textit{a} is a sheriff". $\psi$ = $(\psi_1 \lor \psi_2)$ = "agent
\textit{b} is a villager or godfather or ..." for all 3 roles, except sheriff,
which agent \textit{a} knows he himself is. $\gamma$ = $(\gamma_1 \lor \gamma_2
	\lor .. \lor \gamma_{n-1}) $ = "agent \textit{c} is a villager or godfather
or...", and so on for each agent. \\ In our simulation, we begin the game, by
loading the game-configuration and generate the worlds $W$ by creating all
possible worlds, in accordance with \cref{alg:StartGame}. A round then starts
as described by \cref{alg:dayPhase} with communicative actions, where agent
\textit{b} says "I am villager", represented for agent \textit{a} by $((p \land
	\neg q) \lor (\neg p \land q))$, where p is "agent \textit{b} is a villager and
is telling the truth", and q is "agent b is not a villager and is lying". The
expression essentially represents that agent \textit{a} does not yet know
whether agent \textit{b}'s statement is true or false. But, by inference he may
deduce that if what agent \textit{b} says is correct, then it is also true that
they are the mentioned role. This is represented by $(p \land \neg q)
	\rightarrow \psi_1$ where $\psi_1$ is agent \textit{b} is a villager, and
oppositely $(\neg p \land q) \rightarrow \neg \psi_1$ where $\neg \psi_1$ is
agent \textit{b} is not a villager. \\ Agent \textit{c} also says that they are
a villager, represented by $((g \land \neg h) \lor (\neg g \land h))$ in a
similar manner.

When the agents communicate this, they choose which action to take, based on
\cref{alg:communicate}, which itself invokes \cref{alg:GetHighInfoGainP},
\cref{alg:GetNonAccused}, and \cref{alg:GetMostContradictory}. This whole
sequence of communications can be expressed by the following formula:

\begin{align}
	\begin{split}
		G = K_a(\varphi \land \psi \land\gamma)\land                        \\
		p \rightarrow \psi_1 \land \neg p \rightarrow \neg \psi_1 \land     \\
		q \rightarrow \gamma_1 \land \neg q \rightarrow \neg \gamma_1 \land \\
		((p \land \neg q) \lor (\neg p \land q)) \land                      \\
		((g \land \neg h) \lor (\neg g \land h))
		\label{eq:7}
	\end{split}
\end{align}
These proporsitions are informally represented by the marks added by \cref{alg:updatePossibleWorlds}.\\
When the night phase comes, agent \textit{a}, who is the sheriff, will use \cref{alg:nightPhase} with agent \textit{b} as the target. This results in agent \textit{a} knowing the faction of agent \textit{b}. Agent
\textit{a} will come to know that agent \textit{b} is a member of the mafia.
Directly proving $\neg p$, since the role of a villager belongs to the town
faction, not the mafia faction. This action also excludes all active world from
the sheriff's knowledge, where agent \textit{b} is a member of the town,
leaving him only with world wherein he is a member of the mafia, which in this
case means he must be the godfather. During the same night, agent \textit{b}
targets, agent \textit{c}, killing him. The killing is then publicly announced
during the morning phase, along with the role of the victim. Recall from
\cref{eq:6}, we can say $\varphi_1!G\land\varphi_1$, which informally is that
everything before the public announcement, and the announcement itself, is now
applicable. As previously mentioned, this is performed by using algorithm
\ref{alg:updatePossibleWorlds}, which iterates over all agents and all their
respective worlds, and updates the information. Now recall back to \cref{eq:7},
after the newly acquired information of both $\neg \psi_1$, the knowledge that
agent \textit{b} is not a villager, and $\gamma_2$, the role of agent
\textit{c} we can now simplify the previous expression, which results in:
\begin{align}
	G' = K_a(\varphi \land \neg \psi_1 \land \gamma_1) \land \neg p \land q
\end{align}
Furthermore, as agent \textit{a} began with the knowledge of agent \textit{b}
being either a godfather, or a villager, and they now know that they are not a
villager, they can infer that they must be a godfather, since that is the only
world that has not been excluded, based on the knowledge they gained during the
night. \\
The game then continues with a new communication phase, and the game goes on,
for a short while longer, until the godfather has been voted out.

\section{Results}\label{sec:results}
The original rules for the mafia game excluded roles altogether, instead 
preferring the simplest version of the game consisting of two teams, the mafia 
and the town. Using these rules, it was recommended to have 1 mafia member per. 
3 players, leading to a game setup of 3 mafia members and 7 town 
members\cite{MafiaRules}. Using this game setup in our simulation leads to the 
mafia having a clear advantage, as can be seen on figure 
\ref{fig:VariousSimulations}. The is likely due to the expanded killing 
capabilities in the form of the mafioso and the godfather, relative to a 
limited and a expansion of the town capabilities, in the form of the sheriff 
and the doctor. \\
Should one instead reduce the number of mafia members to 2, and as a result 
increase the number of town members to 8, the win-rates for the two teams 
equalize significantly more, but still being to the mafias favour (MGSD \& 
MGS).\\
Getting to the subject of this report: How a mafia fares against a powerful 
town\footnote{A powerful town being defined as one that uses all roles 
from 
appendix \ref{app:A}.}, one can look at all of the columns after the "Powerful 
Town" label on the figure (\ref{fig:VariousSimulations}). The only consistent 
feature of the mafia teams represented by these columns are the presence of 1 
godfather on each team. They display that, against a powerful town, many 
configurations of mafia-roles are quite balanced, resulting in mostly a 50/50 
win-rate for each team. The main exceptions for this is in regards to 
killing-heavy compositions the most significant of which being MMG, with a 
win-rate of 74\% for the mafia. Furthermore, it seems that one of the only 
roles that can somewhat replace a mafioso in mixed-roles compositions in the 
consigliere, as such compositions (CgCtG \& CgBG) fare comparably to ones 
including a mafioso. \\
Below are all of the results:
\begin{figure}[H]
    \includegraphics[width=1\linewidth]{figures/Winrates}
    \caption{\\Graph of the win-rate of various simulated game compositions.\\
        There were 10 players in each simulation.\\
        Each simulation was 100 games.
        Simulation names are related to team composition, based on role
        abbreviations in appendix \ref{app:A}.\\
        The first 4 runs labelled \textit{Villagers} means that all roles not
        mentioned in the abbreviation are replaced with villagers.\\
        The next runs labelled \textit{Powerful Town} means that all roles not
        mentioned in the abbreviation are replaced with	one of each town role.}
    \label{fig:VariousSimulations}
\end{figure}
\vspace{-5px}Based on this graph we can deduce that mafiosi and
consiglieri are more
impactful for the mafia than both consorts and blackmailers. But it also seems
that the varied team consisting of one mafioso, consigliere, and godfather,
performs marginally better than other configurations. \\
The results indicate that when playing against a powerful town, it is more
important for the mafia to be able to continually kill town members, and to be
able to quickly find the most powerful roles of the town, in order to eliminate
them, than it is to deny actions and communication to the town. \\
It would be interesting to see which of the non-killing mafia roles perform
best when they are on their own, and which composition of them perform best.
This type of game removes the mafias ability to kill during the night, while
they retain their knowledge of one another and their respective abilities to
gain information, limit nightly actions, and limit communicative actions. Below
are the results of such games:
\begin{figure}[H]
    \includegraphics[width=1\linewidth]{figures/Winrates_NonKilling}
    \caption{\\Graph of the win-rate of various simulated game compositions
        with only non-killing mafia roles.\\
        There were 10 players in each simulation.\\
        Each simulation was 100 games.
        Simulation names are related to team composition, based on role
        abbreviations in appendix \ref{app:A}.\\
        All of the runs were against the previously explained powerful town.}
    \label{fig:VariousSimulationsNonKilling}
\end{figure}
\vspace{-5px} As seen on the above graph we can once again conclude that
consiglieri are more impactful than either consorts or blackmailers \ref{fig:VariousSimulationsNonKilling}. This may
be due to their superior abilities in finding the sheriff, enabling them to
collectively vote them out. It also seems that no mixed-combination of
non-killing mafia roles can compete with the pure-consiglieri team.

\input{sections/discussion.tex}
\input{sections/conclusion.tex}
\section{Future Works}\label{sec:future-works}
Should research into this area be continued, one might suggest beginning by
performing alterations in the following areas:
\begin{enumerate}
	\itemsep0px 
	\item More roles such as: framer, janitor, bodyguard, lookout, etc.
	\item More factions such as: neutral roles, werewolves, witches, vampires, etc.
	\item Re-factorization of the code into building knowledge from zero, instead of
	      pruning knowledge from all possibilities. This would result in improved run
	      times and smaller memory usage.
	\item Possible representation of knowledge in the code as Binary Decision Diagrams or
	      Logical formulae. This would further improve run times and memory usage.
	\item Clearer choices between accuse and inquire communicative actions, as the game
	      currently chooses these actions randomly despite their potentially inequivalent
	      outcomes.
	\item Clearer voting strategies for whom the town should target first, as their
	      highest priority, as currently they vote randomly within mafia members.
	\item Add method for Consigliere to share their gained information with the mafia.
	\item Fine-tune information gained by investigator.
\end{enumerate}
\section*{Special Thanks}
As a final note, we would like to thank our supervisor, Simonas Saltenis, of the Department of Computer Science at Aalborg University. Without him this project would not have been possible. 
\clearpage
\appendix
\makeatletter
\def\@seccntformat#1{\appendixname\ \csname the#1\endcsname: }
\makeatother

\section{Appendix A}\label{app:A}
\begin{center}
	\textbf{Game Roles}
\end{center}
The following is a detailed description of the different roles and their unique
nightly actions:

\textbf{Roles allied to the town:}
\begin{enumerate}
	\item\textbf{Villager (V)} is a  role with no special actions. Typically
	      given
	      to the majority of the town.
	\item\textbf{Sheriff (S)} must choose one player each night and privately
	      gain
	      information regarding their faction.
	\item\textbf{Investigator (I)} must choose one player each night and
	      privately
	      gain information regarding a selection of roles that that player may have,
	      one of them being true.
	\item\textbf{Doctor (D)} must choose one player each night and protect them
	      from any harm that may come to them during the night.
	\item\textbf{Escort (E)} must choose one player each night and prevent them
	      from performing any actions during the night phase.
	\item\textbf{Veteran (Ve)} may choose to go on alert during the night
	      phase, if
	      they do so, then anyone targeting them at night will die.
	\item\textbf{Vigilante (Vi)} may choose to kill any one player during the
	      night
	      phase, if that player is revealed to be a member of the town during the
	      morning phase, the vigilante loses their killing ability, and will die the
	      following morning.
\end{enumerate}
\textbf{Roles allied to the mafia:}
\begin{enumerate}
	\item\textbf{Godfather (G)} must choose one player each night who will die.
	\item\textbf{Mafioso (M)} performs the killing on the Godfathers behalf. If
	      the
	      Godfather is targeted by an escort, then the mafioso may independently
	      choose who to kill. If the mafioso is targeted by the escort, then the
	      Godfather kills their chosen target themselves.
	\item\textbf{Consigliere (Cg)} must choose one player each night and
	      privately
	      gain information regarding their role.
	\item\textbf{Consort (Ct)} must choose one player each night and prevent
	      them
	      from performing any actions during the night phase.
	\item\textbf{Framer (F)} must choose one player each night to frame. A
	      framed
	      player appears to be a member of the mafia if chosen by the Sheriff on this
	      night.
	\item\textbf{Blackmailer (B)} must choose one player each night to
	      blackmail. A
	      blackmailed player can only choose the “do not say anything” communicative
	      action.
\end{enumerate}
\section*{Appendix B}\label{app:B}
\begin{center}
	\textbf{Semantic Examples}
\end{center}
\textbf{(1)} \\
If one inquires another the answer, gamma, will could be of any of the 
following forms "I belive that person 1 is a member of the mafia.", "i believe 
that person 1 is not a member of the mafia.", or "I believe that person 1 is a 
member of the mafia, and person 2 is a member of the town."  \\ \\
\textbf{(2)} \\
No more oracle? \\ \\
\textbf{(3)} \\
This describes every individual agent's worldview. It consists of some, 
non-empty set of possible worlds, in which some are epistemicly equivalent to 
each other, due to the agent's limited knowledge. These worlds contain facts, 
such as a person 1-2-3 being role x-y-z. Furthermore, agents implicitly know 
all inference rules, such as, "if no one was killed during the night, then both 
the mafioso and the godfather were targeted by escorts." Using this rule, an 
escort may be able to infer that, if that was the result of the night phase, 
then their target during the night is very likely to be one of those roles.  \\ 
\\
\textbf{(11), (12), (13)} \\
Simply states that a given agent can now update their worldview when given a 
proposition. The new worldview is a subset of the original worldview, in which 
all worlds in which the proposition does not hold are excluded. 


\clearpage
% Bibliography
\printbibliography[heading=bibintoc, title=Bibliography]
\label{bib:mybiblio}

\end{document}
