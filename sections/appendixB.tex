\section*{Appendix B}\label{app:B}
\begin{center}
	\textbf{Semantic Examples}
\end{center}
\textbf{(1)} \\
If one inquires another the answer, gamma, will could be of any of the 
following forms "I belive that person 1 is a member of the mafia.", "i believe 
that person 1 is not a member of the mafia.", or "I believe that person 1 is a 
member of the mafia, and person 2 is a member of the town."  \\ \\
\textbf{(2)} \\
No more oracle? \\ \\
\textbf{(3)} \\
This describes every individual agent's worldview. It consists of some, 
non-empty set of possible worlds, in which some are epistemicly equivalent to 
each other, due to the agent's limited knowledge. These worlds contain facts, 
such as a person 1-2-3 being role x-y-z. Furthermore, agents implicitly know 
all inference rules, such as, "if no one was killed during the night, then both 
the mafioso and the godfather were targeted by escorts." Using this rule, an 
escort may be able to infer that, if that was the result of the night phase, 
then their target during the night is very likely to be one of those roles.  \\ 
\\
\textbf{(11), (12), (13)} \\
Simply states that a given agent can now update their worldview when given a 
proposition. The new worldview is a subset of the original worldview, in which 
all worlds in which the proposition does not hold are excluded. 
