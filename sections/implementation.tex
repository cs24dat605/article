\section{Implementation}\label{sec:implementation}
With the theory explained, we can now explore our implementation, which aims to
be as true to our formal description as possible. We will start by laying the
foundation of the most pivotal data structures, followed by a walk-through of
the implementation of each phase of the game.

\subsection{Foundational Data Structures}\label{sec:foundationalDataStructures}
To provide some context for the upcoming sections, we will first describe the
data structures they are based upon.

A game consists of some $n$ number of agents, wherein each agent has $m$ amount
of worlds. $m$ is defined as described in \cref{sec:TheKnowledgeOfAgents}. A
world, often referred to as a possible world, describes a possible combination
of roles and agents. Each world has a list of roles, which represents that
specific world's role composition. This relation is depicted in
\cref{fig:DataStructuresUML}. \renewcommand{\umlfillcolor}{white}
\renewcommand{\umldrawcolor}{blue}
\begin{figure}[H]
	\centering
	\scalebox{0.8}{
		\begin{tikzpicture}
			\begin{class}[text width= 3cm]{Game}{-0.5, -0.63}
				\attribute{agents : list}
				\operation{runDayPhase()}
				\operation{runNightPhase()}
				\operation{checkFinished()}
			\end{class}
			\begin{class}[text width=4.5cm]{Agent}{5,0}
				\attribute{worlds : list}
				\attribute{communicationLog : list}
				\attribute{role : string}
				\attribute{isAlive : boolean}
				\operation{updateWorlds(information)}
				\operation{nightAction()}
				\operation{communication()}
			\end{class}
			\begin{class}[text width=4cm]{World}{-0.5,-4.85}
				\attribute{possibleAgents: list}
				\attribute{marks : integer}
				\attribute{isActive : boolean}
				\attribute{isPrivateActive : boolean}
			\end{class}
			\begin{class}[text width=3.5cm]{PossibleAgent}{5,-5}
				\attribute{actualAgent: agent}
				\attribute{isAlive : boolean}
				\attribute{possibleRole : string}
			\end{class}
			\unidirectionalAssociation{Game}{}{1..n}{Agent}{}{}
			\unidirectionalAssociation{Agent}{}{1..m}{World}{}{}
			\unidirectionalAssociation{World}{1..n}{}{PossibleAgent}{}{}
			\unidirectionalAssociation{PossibleAgent}{*..1}{}{Agent}{}{}
		\end{tikzpicture}}
	\vspace*{3mm}
	\caption{Simplified UML diagram of data structures, where n is the amount of agents and m is the amount of worlds for each agent.}
	\label{fig:DataStructuresUML}
\end{figure}
We argue this encapsulates the model both described in \cref{sec:Modelling}, and the formalisation in
\cref{sec:DynamicInquiryLanguage}, equation \cref{eq:3}, most noteworthy is the relation between $W$ and \textbf{Ag}. The individual elements of \cref{fig:DataStructuresUML} will be explained further in the following sections.
\subsubsection{The Knowledge of Agents}\label{sec:TheKnowledgeOfAgents}
The knowledge and logical propositions of agents has been chosen to be
represented with a truth-table approach, which refers to the possible worlds
which store all possible pairings of agents and roles. The following equation
shows the size of the list given agent count $A$ and the number of agents with
role $R$, this expression also assumes that agents that are not assigned a
role, will become villagers.
\begin{equation}
	W: A,R \in  \mathbb{N} \mapsto \frac{A!}{R!(A-R)!}\
\end{equation}

For each subsequent role, $R_1, R_2, .., R_n$ that is added, the complexity
expands to:
\begin{equation}
	\begin{gathered}
		W: A, R_0 = 0, R_1, ..., R_n, n\in  \mathbb{N} \mapsto \\
		\frac{\prod\limits_{i=1}^{n}(A-\sum\limits_{j=1}^{i}R_{j-1})!}{\prod\limits_{i=1}^{n}R_i!(A-\sum\limits_{j=1}^{i}R_j)!}
		\label{eq:numofworlds}
	\end{gathered}\footnote{If the number of agents equals the
		number of roles, then the formula collapses into A!.}
\end{equation}

Each agent excludes possible-worlds that are not in accordance with the role
that they have been assigned.\\ \textit{For example}: Agent \textit{a}, if
given the role of sheriff, will exclude all possible-worlds where their role is
not sheriff, while another agent \textit{b} may keep some of those
possible-worlds, since they do not know the role of \textit{a}.\\ In our
implementation, this means that for agent \textit{a} the worlds, where they are
not a sheriff are privately inactive. Furthermore, each agent also contains a
list of who have made what claims, and with this, they can extrapolate whether
that claim is true or false in a given possible-world. This is represented by a
list of communicative actions, and the amount of lies of agents, is stored in
the possible worlds as \textit{marks}. If a world has many marks, it implies
that many agents have lied if this is the true world, which would be unlikely.
It is also important to remember, that we keep track of who contributed to each
mark in a given world. This will be explained further in the following
sections.

\subsection{Updating Knowledge}\label{sec:UpdatingKnowledge}
When a new fact is revealed, such as when an agent dies and their role is
revealed, all agents will immediately exclude all possible-worlds in which the
dead agent is anything other than their revealed role. Implementation wise,
this means that the worlds which are not in accordance with the new information
are set as inactive\footnote{Practically this is equivalent to them being
	deleted}. This corresponds to the model update defined in \cref{eq:5}. An agent
dying is the way that most roles learn new facts. \\ Depending on the role that
died, all agents update the marks that the dead agent contributed to.\\
\textit{For example}: A sheriff has a nightly action which allows them to gain
more information than regular villagers. Therefore, the marks that a sheriff
has contributed to, should be weighed more heavily than those of other roles.
To emulate this, other agents add an additional mark to any marks that were
contributed to by the sheriff, to further denote the importance of their
opinions. \\ Conversely, should the dead agent be revealed to be a member of
the mafia, all agents will detract a mark from the marks that the dead agent
contributed to. Due to mafia member have incentive to deceive the other agents.

\subsection{Beginning of the Game}\label{sec:beginningOfTheGame}
With the foundational data structures explained, we can begin the game, which
is done in accordance with the pseudocode in \cref{alg:StartGame}.
\begin{algorithm}[H]
	\caption{StartGame}
	\begin{algorithmic}[1]
		\Function{StartGame}{}
		\State loadGameConfiguration()
		\State generateWorlds()

		\While{!gameFinished}
		\State dayPhase()
		\State nightPhase()
		\State checkFinished()
		\EndWhile
		\EndFunction
	\end{algorithmic}\label{alg:StartGame}
\end{algorithm}
\setcounter{algorithmcaption}{0}
\captionof{algorithmcaption}{Pseudocode for starting and running the
	game.}
The game begins by loading the role-configuration for that specific game, and
then generating the worlds for all agents. It then goes through the day- and
night phases, followed by checking if either team has won.

\subsection{Day Phase}\label{sec:dayPhase}
The day phase is split into two sub-phases: communication and voting. These
phases are performed in order, and controlled by \cref{alg:dayPhase}.
\begin{algorithm}[H]
	\caption{Day phase}
	\begin{algorithmic}[1]
		\Function{dayPhase}{}
		\ForAll{a $in$ game.Agents}
		\State a.communicate()
		\EndFor
		\ForAll{a $in$ game.Agents}
		\State votes.append(voting(a))
		\EndFor

		\State killMostVoted(votes)
		\EndFunction
	\end{algorithmic}\label{alg:dayPhase}
\end{algorithm}
\setcounter{algorithmcaption}{1}
\captionof{algorithmcaption}{The order of sub-phases within the day phase.}
First all agents communicate, then they all vote. The voting function, returns the target agent, which the calling agent, believes is the most suspicious and adds it to the \textit{votes} list.

\subsubsection{Communication}\label{sec:communication}
The communication sub-phase is where agents share information with each other
based on the marks in their active worlds. Communicative actions are the main
way to contribute to marks in a given possible-world. The way that a given
agent chooses who to target with their communicative action is chosen in
accordance with \cref{alg:GetHighInfoGainP}.
\begin{algorithm}[H]
	\caption{GetHighInfoGainA(me, tWorlds)}
	\begin{algorithmic}[1]
		\If{nonAccused.Count $>$ 0 OR Game.Round = 0}
		\State $(a, wID)$ $\gets$ GetNonAccused(me)
		\If{wID != -1}
		\State \Return (a, wID)
		\EndIf
		\EndIf

		\State (a, wID) $\gets$ GetMostContradictory(me, tWorlds)
		\If{wID != -1}
		\State \Return (a, wID)
		\EndIf

		\State w $\gets$ random world from me.Worlds \State
		a $\gets$ random agent from w.PossAgents \State \Return (a, index of w in
		me.Worlds)
	\end{algorithmic}\label{alg:GetHighInfoGainP}
\end{algorithm}
\setcounter{algorithmcaption}{2}
\captionof{algorithmcaption}{The base algorithms that handles the communicative logic.}
Where the parameter \textit{me} refers to the agent deciding an action, and \textit{tWorlds} is the top 1\% least marked worlds for \textit{me}. To get the highest information gain, \cref{alg:GetHighInfoGainP} returns an
agent based on different factors.
First, it prioritizes non-accused agents as shown in
\cref{alg:GetNonAccused}.
Then finds the most contradictory information as shown in
\cref{alg:GetMostContradictory}. We do this, to first gain some information on every player, and then target the one with most contradictions. \\
\begin{algorithm}[H]
	\caption{GetNonAccused(me)}
	\begin{algorithmic}[1]
		\If{nonAccused.Count = 0 AND Game.Round = 0}
		\State nonAccused $\gets$ all agents in me.Worlds[0]
		\EndIf

		\ForAll{communication in me.CommunicationLog}
		\State nonAccused.Remove(communication.target)
		\EndFor

		\State nonAccused.Remove(me)

		\If{nonAccused.Count != 0}
		\State \Return nonAccused[random]
		\EndIf

		\State \Return (null, -1)
	\end{algorithmic}\label{alg:GetNonAccused}
\end{algorithm}
\setcounter{algorithmcaption}{3}
\captionof{algorithmcaption}{Returns an agent that haven't been accused yet. nonAccused is a local list of agents which have not yet been targeted.}
Each agent initialises a list of all other agents at the
start of the first communication phase. Whenever an agent targets another agent
with a communicative action, they remove the target agent from their individual
list. The target will also be removed from the other agents lists, before the target for their communicative action is decided. When all agents have been targeted with one communicative action, some knowledge on all agents is available.They will use to determine
the new best target for their communicative
actions. This is done in the next step of \cref{alg:GetHighInfoGainP},
by invoking \cref{alg:GetMostContradictory}.
\begin{algorithm}[H]
	\caption{GetMostContradictory(me, tWorlds)}
	\begin{algorithmic}[1]
		\State roleCounts $\gets$ getRoleCounts(tWorlds)

		\ForAll{w in tWorlds}
		\ForAll{a in w.PossAgents where a != me AND a is alive}
		\State highest $\gets$ roleCounts[a].Max()
		\State secHighest $\gets$ roleCounts[a].SecondMax()

		\State contraScore $\gets$ secHighest / highest
		\If{contraScore $>$ maxScore}
		\State mostContradictory $\gets$ a
		\State maxScore $\gets$ contraScore
		\State wID $\gets$ index of w
		\EndIf
		\EndFor
		\EndFor
		\State \Return (mostContradictory, wID)
	\end{algorithmic}\label{alg:GetMostContradictory}
\end{algorithm}
\setcounter{algorithmcaption}{4}
\captionof{algorithmcaption}{Returns an agent with the most conflicting role claims. \textit{roleCounts} is the amount of different roles, an agent is in tWorlds}
This algorithm outputs the agent which is the most suspicious due to being
suspected of being the highest difference of roles in the top 1\% of active possible-worlds.\\
This whole process is described in the following high-level algorithm
\ref{alg:communicate} for how one agent handles their turn in the communication
sub-phase.
\begin{algorithm}[H]
	\caption{Communication}
	\begin{algorithmic}[1]
		\Function{communicate}{}
		\State world $\gets$ getMostPlausibleWorld()
		\State action $\gets$ self.getAction(world)

		\If{action is inquire or accuse}
		\State a $\gets$ getHighestInfoGainA(me, tWorlds)
		\State question $\gets$ getHighestInfoQuestion(a)
		\State commAction $\gets$ me.ask(a, question)
		\ElsIf{action is defend}
		\State commAction $\gets$ me.claimRole()
		\EndIf

		\State updatePossibleWorlds(commAction)
		\EndFunction
	\end{algorithmic}\label{alg:communicate}
\end{algorithm}
\setcounter{algorithmcaption}{5}
\captionof{algorithmcaption}{High-level pseudocode of an agents turn during the
	communication sub-phase.}
It describes how an agent's turn consists of finding their most plausible world,
deciding what actions to take, choosing their target, as described above, and
then updating their worlds' marks according to the outcome of the action.
The updating of their worlds is performed in accordance with
\cref{alg:updatePossibleWorlds}, seen below.
\begin{algorithm}[H]
	\caption{Update possible worlds}
	\begin{algorithmic}[1]
		\Function{updatePossWorlds}{information}
		\ForAll{a $in$ game.Agents}
		\ForAll{world $in$ a.Worlds}
		\State world.UpdateMarks(information)
		\EndFor
		\EndFor
		\EndFunction
	\end{algorithmic}\label{alg:updatePossibleWorlds}
\end{algorithm}
\setcounter{algorithmcaption}{6}
\captionof{algorithmcaption}{The updating of possible worlds. \textit{information} refers to either a communicative action or when a player dies.}
Whenever an agent gets an answer for their communicative action, or when other information is gained, all agents update their
world view by invoking \cref{alg:updatePossibleWorlds}. The \textit{UpdateMarks} function reviews the new information, and checking if the information is true or false in that world. If the information is false then a mark is added otherwise nothing happens.\\

The phase ends when a variable amount of communicative actions have been
performed by each agent, which, in this implementation, has been arbitrarily
set to 3.

\subsubsection{Voting}\label{sec:voting}
When communication and the updating of possible-worlds has finished, the game
moves on to the voting phase. Each agent chooses who to vote for, in a very
similar manner to the choice of target for their communicative action
previously shown in \cref{alg:GetHighInfoGainP}. They chose the active world
with the least marks, i.e. the one they find to be the most likely. If there
are multiple, then they pick randomly between them. They then select a random
member of the mafia from that world, and vote for that agent.

\subsection{Night Phase}\label{sec:nightPhase}
The night phase is also broken down into two sub-phases: the night phase and
the nightly actions phase.

\subsubsection{Nightly
	Actions}\label{sec:nightPhaseNightlyActions}
Nightly actions are performed during the night phase, and are tailored towards
each role. For each role, we define their individual most likely world $W$, as
the world with the least amount of marks for them. \\ An example of how a
nightly action is performed can be seen in \cref{alg:nightPhase}.
\begin{algorithm}[H]
	\caption{Night action}
	\begin{algorithmic}[1]
		\Function{performNightAction}{}
		\If{me.role == Sheriff}
		\State world $\gets$ getMostPlausibleWorld()

		\State susAgents $\gets$ getSortedPossMafAgents(world)
		\State susAgents.remove(me.previousTargets)

		\State me.actResult $\gets$ interrogate(susAgents[0])
		\EndIf
		\EndFunction

		\Statex
	\end{algorithmic}\label{alg:nightPhase}
\end{algorithm}
\setcounter{algorithmcaption}{7}
\captionof{algorithmcaption}{An example of a nightly action performed by the
	sheriff. \textit{interrogate} sets all worlds that are not in accordance with the new information to privately-inactive}
The sheriff finds who the mafia members are in $W$, and then targets one of
them, which he has not targeted before. Other agents perform their actions
similarly, such as the godfather trying to kill whomever is the sheriff in $W$.
The doctor tries to save whomever is the sheriff in $W$, etc.

\subsubsection{Morning Phase}\label{sec:morningPhase}
The morning phase begins after the night phase. Its only purpose is to publicly
announce the public effects of the nightly actions, and then make all agent
update their world-views, by using the previously discussed
\cref{alg:updatePossibleWorlds}. However, in the implementation facts are
acquired as soon as an action has occurred. This change does not affect the
game, as all agents already have declared which night action they want to
take.\\ As facts are revealed, possible-worlds are de-activated, meaning that
one may slowly infer the roles of other agents. Furthermore, as people die,
marks are added or altered based on how trustworthy the given victim's role
was. \\ It is important to note, that while the semantics include inference in
the form of propositions, and communicative actions can be expressed in the
propositional language, this is not included in our implementation. However,
since we include marks, and our decisions are based on these, we denote this as
weak inference of the communicative actions.

\subsection{Detailed run through}\label{sec:ARoundOfTheGame}
A round of our modelled game can be expressed in the dynamic inquiry language.
In a model defined as \cref{eq:3}, we let there be 10 agents \textbf{Ag} = \{a,
b, c, ..., j\}, where \textit{a} is a sheriff, \textit{b} is a godfather, and
the rest are villagers. This round is viewed from the sheriffs perspective, so
he does not know the roles of the other agents, only the amount of each role
that is in the game, i.e. 1 sheriff, 1 godfather, and 8 villagers. He then has
a set of propositions \{$\varphi$, $\psi$, $\chi$, $...$\} where $\varphi$ =
"agent \textit{a} is a sheriff". $\psi$ = $(\psi_1 \lor \psi_2 \lor ... )$ =
"agent \textit{x} is a villager or godfather or ..." for all 3 roles, except
sheriff, which agent \textit{a} knows he himself is, for each agent $x \in
	\text{\textbf{Ag}}$. \\ In our simulation, we begin the game, by loading the
game-configuration and generate the worlds $W$ by creating all possible worlds,
in accordance with \cref{alg:StartGame}. A round then starts as described by
\cref{alg:dayPhase} with communicative actions, where agent \textit{b} says "I
am villager", represented for agent \textit{a} by $((p \land \neg q) \lor (\neg
	p \land q))$, where p is "agent \textit{b} is a villager and is telling the
truth", and q is "agent b is not a villager and is lying". The expression
essentially represents that agent \textit{a} does not yet know whether agent
\textit{b}'s statement is true or false. But, by inference he may deduce that
if what agent \textit{b} says is correct, then it is also true that they are
the mentioned role. This is represented by $(p \land \neg q) \rightarrow
	\psi_1$ where $\psi_1$ is agent \textit{b} is a villager, and oppositely $(\neg
	p \land q) \rightarrow \neg \psi_1$ where $\neg \psi_1$ is agent \textit{b} is
not a villager. \\ Agent \textit{c} also says that they are a villager,
represented by $((g \land \neg h) \lor (\neg g \land h))$ in a similar manner.

When the agents communicate this, they choose which action to take, based on
\cref{alg:communicate}, which itself invokes \cref{alg:GetHighInfoGainP},
\cref{alg:GetNonAccused}, and \cref{alg:GetMostContradictory}. This whole
sequence of communications can be expressed by the following formula:

\begin{align}
	\begin{split}
		G = K_a(\varphi \land \psi \land\chi)\land                      \\
		p \rightarrow \psi_1 \land \neg p \rightarrow \neg \psi_1 \land \\
		q \rightarrow \chi_1 \land \neg q \rightarrow \neg \chi_1 \land \\
		((p \land \neg q) \lor (\neg p \land q)) \land                  \\
		((g \land \neg h) \lor (\neg g \land h))
		\label{eq:7}
	\end{split}
\end{align}
These proporsitions are informally represented by the marks added by \cref{alg:updatePossibleWorlds}. After communication have ceased the voting phase begins. For the simplicity of this example we assume a tie in the voting phase resulting in no eliminations.  \\
When the night phase comes, agent \textit{a}, who is the sheriff, will use \cref{alg:nightPhase} with agent \textit{b} as the target. This results in agent \textit{a} knowing the faction of agent \textit{b}. Agent
\textit{a} will come to know that agent \textit{b} is a member of the mafia.
Directly proving $\neg p$, since the role of a villager belongs to the town
faction, not the mafia faction. This action also excludes all active world from
the sheriff's knowledge, where agent \textit{b} is a member of the town,
leaving him only with worlds wherein agent \textit{b} is a member of the mafia, which in this
case means he must be the godfather. During the same night, agent \textit{b}
targets agent \textit{c}, eliminating them. The elimination is then publicly announced
during the morning phase, along with the role of the victim. Recall from
\cref{eq:6}, we can say $\varphi_1!G\land\varphi_1$, which informally is that
everything before the public announcement, and the announcement itself, is now
applicable. As previously mentioned, this is performed by using algorithm
\ref{alg:updatePossibleWorlds}, which iterates over all agents and all their
respective worlds, and updates the information. Now recall back to \cref{eq:7},
after the newly acquired information of both $\neg \psi_1$, the knowledge that
agent \textit{b} is not a villager, and $\chi_2$, the role of agent
\textit{c} we can now simplify the previous expression, which results in:
\begin{align}
	G' = K_a(\varphi \land \neg \psi_1 \land \chi_1) \land \neg p \land q
\end{align}
Furthermore, as agent \textit{a} began with the knowledge of agent \textit{b}
being either a godfather, or a villager, and they now know that they are not a
villager, they can infer that they must be a godfather, since that is the only
world that has not been excluded, based on the knowledge they gained during the
night. \\
The game then continues with a new communication phase, and the game goes on.
