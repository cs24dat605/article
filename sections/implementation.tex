\section{Implementation}\label{sec:implementation}
With the theory explained, we can now explore our implementation, which aims to
be as true to reality as possible, while being equivalent to the theory. At the
end of this section, we will go through an example round of our implementation,
but before doing that, we will need to build the foundation of it.
\subsection{The Knowledge of Agents}\label{TheKnowledgeOfAgents}
The knowledge and logical propositions of agents has been chosen to be
represented with a truth-table approach. Every agent has their own list of
possible-worlds, in which is contained every single combination of players and
roles. This means, that the more players and roles there are, the longer this
list becomes. The following equation shows how the list grows given player
count P and one role R1.
\begin{gather}
	W(P,R1) := \frac{P!}{R1!(P-R1)!}\
\end{gather}
For each subsequent role that is added, the formula is expanded with
\begin{equation}
	\begin{gathered}
		W(P, R1, R2) := \\
		\frac{P!\cdot(P-R1)!}{R1!\cdot(P-R1)!\cdot
			R2!\cdot(P-R1-R2)}\
	\end{gathered}\footnote{If the number of players equals the
		number of roles, then the formula collapses into P!.}
\end{equation}
Each agent excludes possible-worlds that are not in
accordance with the role that they have been given.\\
For example: agent, \textit{a}, if given the role of sheriff, will exclude all
possible-worlds where their role is not sheriff, while another agent \textit{b} may
keep some of those possible-worlds, since they do not know the role of \textit{a}.\\
Furthermore, each column also contains data on who have made what claims, and
whether that claim is true or false in this given possible-world. This is represented
by an array for each possible-world, which keeps track of the lies of players.Going forward, these lies will be known as
\textit{marks}, and what is important to know is that, if a world has many
marks, then it means that many players have lied if this world is true. It is
also important to remember, that we keep track of who contributed to each mark
in a given world. This will be explained further in the following sections.
\subsection{Updating Knowledge}\label{UpdatingKnowledge}
When a new fact is revealed, such as when a player dies and their role is
revealed, all players will immediately exclude all possible-worlds in which the
dead player is anything other than their revealed role. A player dying is the
way that most roles learn new facts.\\ Furthermore, depending on the role that
died, all players update the marks that the dead player contributed to.\\ For
example: A sheriff has a nightly action which allows them to gain more
information that regular villagers can. Therefore, the marks that a sheriff has
contributed to, should be weighed more heavily than those of other roles. To
emulate this other players add an additional mark, to any marks that were
contributed to by the sheriff, to further denote the importance of their
opinions. \\ Conversely, should the dead player be revealed to be a member of
the mafia, all players will detract a mark from the marks that the dead player
contributed to. This is because the mafia member may have tried to deceive the
other players
\subsection{Communicative Actions}\label{CommunicativeActions}
Communicative actions are the main way to contribute to marks in a given
possible-world. Let us assume that agent \textit{a} chooses the following
communicative action: "\textit{\ref{Com}. Claim to have a certain role.}" and
states, truthfully, "\textit{I am a Sheriff}". This will prompt all other
players to add a mark to all possible-worlds in which \textit{a} is not a
sheriff. The next player then chooses a communicative action, leading to a
similar course of actions from the other players. \\ The result of all players
communicating will be that players have added a significant amount of marks to
their individual worlds. This means that they will each now have some worlds
which are more likely to be true than others, since we assume that most people
tell the truth, due to townspeople having no incentive to lie. This ranking of
worlds will be useful going forward.\\ Almost all choices in the simulation is
in some way related to the amount of marks in their active worlds. The
following pseudo code relates to how an agent chooses what communicative action
to perform: \\
\begin{algorithm}[H]
	\caption{GetHighInfoGainP(me, tWorlds)}
	\begin{algorithmic}
		\If{nonAccused.Count $>$ 0 OR Game.Round = 0}
		\State (p, wID) $\gets$ GetNonAccused(me)
		\If{wID != -1}
		\State \Return (p, wID)
		\EndIf
		\EndIf

		\State (p, wID) $\gets$ GetMostContradictory(tWorlds, me)
		\If{wID != -1}
		\State \Return (p, wID)
		\EndIf

		\State w $\gets$ random world from me.PossWorlds
		\State p $\gets$ random player from w.PossPlayers
		\State \Return (p, index of w in me.PossWorlds)
	\end{algorithmic}
\end{algorithm}
\begin{algorithm}[H]
	\caption{GetNonAccused(me)}
	\begin{algorithmic}
		\If{Game.Round != 0 AND nonAccused.Count = 0}
		\State \Return (null, -1)
		\EndIf

		\If{nonAccused.Count = 0 AND Game.Round = 0}
		\State nonAccused $\gets$ all players in me.PossWorlds[0]
		\EndIf

		\ForAll{accusation in me.nonAccu}
		\State me.nonAccu.Remove(accusation.Acusse)
		\EndFor

		\State me.nonAccu.Remove(me)

		\If{nonAccused.Count != 0}
		\State \Return nonAccused[randomI]
		\EndIf

		\State \Return (null, -1)
	\end{algorithmic}
\end{algorithm}
\begin{algorithm}[H]
	\caption{GetMostContradictory(tWorlds, me)}
	\begin{algorithmic}
		\State roleCounts $\gets$ getRoleCounts(tWorlds)

		\ForAll{w in tWorlds}
		\ForAll{p in w.PossPlayers where p != me AND p is alive}
		\State contraScore $\gets$ secondHighest / highest
		\If{contraScore > maxScore}
		\State mostContradictory $\gets$ p
		\State maxScore $\gets$ contraScore
		\State wID $\gets$ index of w
		\EndIf
		\EndFor
		\EndFor
		\State \Return (mostContradictory, wID)
	\end{algorithmic}
\end{algorithm}
The choice of who to target is based on
the belief that any proposition regarding that player will result in the most
amount of worlds being marked as inactive whenever it is proven or disproven.
\subsection{Voting}\label{Voting}
When communication and the addition of marks has finished, the game moves on to
the voting phase. Each player chooses who to vote for, in a very similar manner
to the choice of target for their communicative action. They chose the world
with the least marks, i.e. the one they find to be the most likely. If there
are multiple, then they pick randomly between them. They then select a random
member of the mafia from that world, and then vote for that person.
\subsection{Nightly Actions}\label{NightlyActions}
Nightly actions are a bit more tailored towards each role. For example: The
godfather simply tries to kill whomever is the sheriff in their most likely
world. The doctor tries to save whomever is the sheriff in their most likely
world, and the sheriff checks the world with the least marks, then looks at who
he thinks is the mafia in that world and checks one of them. The sheriff keeps
a list of all players that he has checked, such that he does not check a player
twice.
\subsection{Weak Inference}\label{WeakInference}
While our simulation does not include any actual inference rules, it does mimic
a kind of inference. As facts are revealed, possible-worlds, are removed,
meaning that one may slowly infer the roles of players while not knowing their
actual role. Furthermore, as people die and marks are added or altered based on
how trustworthy the given victims role were, another kind of weak inference is
performed, as players do not base their decisions on worlds that are unlikely.
\subsection{Detailed run through}\label{ARoundOfTheGame}
A round of our modeled game can be expressed in the dynamic inquiry language.
In a model defined as \cref{eq:3}, we let there be 10 agents \textbf{Ag} = \{a,
b, c, ..., j\}, where \textit{a} is a sheriff, \textit{b} is a godfather, and
the rest are villagers. This round is viewed from the sheriffs perspective, so
he does not know the roles of the other players, only the amount of each role
that is in the game, i.e. 1 sheriff, 1 godfather, and 8 villagers. He then has
a set of propositions \{$\varphi$, $\psi$\, $\gamma$, ...\} where $\varphi$ =
"agent \textit{a} is a sheriff". $\psi$ = $(\psi_1 \lor \psi_2 \lor .. \lor
	\psi_{n-1})$ = "agent \textit{b} is a villager or godfather or ..." for all
\textit{10} roles, except sheriff, which agent \textit{a} knows he himself is.
$\gamma$ = $(\gamma_1 \lor \gamma_2 \lor .. \lor \gamma_{n-1}) $ = "agent
\textit{c} is a villager or godfather or...", and so on for each player. \\ In
our simulation, we generate $W$ by creating all possible worlds, based on the
aforementioned propositions, by utilizing the pseudocode function
\lstinline{generateWorlds()} provided in appendix \ref{app:B}. \\ A round then
starts with communicative actions, where agent \textit{b} says "I am villager",
represented for agent \textit{a} by $(p \lor \neg p)$, meaning that agent
\textit{a} does not yet know whether agent \textit{b}'s statement is true or
false. But, by inference he may deduce that if what agent \textit{b} says is
correct, then it is also true that they are the mentioned role. This is
represented by $p \rightarrow \psi_1$ where $\psi_1$ is agent \textit{b} is a
villager, and oppositely $\neg p \rightarrow \neg \psi_1$ where $\neg \psi_1$
is agent \textit{b} is not a villager. \\ Agent \textit{c} also says that they
are a villager, represented by $q$ in a similar manner.

\begin{algorithm}
	\caption{Communication}
	\begin{algorithmic}[1]
		\Function{communicate}{}
		\State world $\gets$ getMostPlausibleWorld()
		\State action $\gets$ self.getAction(world)

		\If{action is inquire or accuse}
		\State p $\gets$ getHighestInfoGainP()
		\State question $\gets$ getHighestInfoQuestion(p)
		\State commAction $\gets$ self.ask(p, question)
		\ElsIf{action is defend}
		\State commAction $\gets$ self.claimRole()
		\EndIf

		\State updatePossibleWorlds(commAction)
		\EndFunction
	\end{algorithmic}
\end{algorithm}\label{lst:communicate}

When the agents communicate this, they choose which action to take, based on
what communicative action will result in the most amount of worlds being marked
as inactive, whenever the proposition provided by the communicative action is
proven or disproven. This whole sequence of the original example can be
expressed by the following formula:

\begin{align}
	\begin{split}
		G = K_a(\varphi \land \psi \land\gamma)\land                        \\
		p \rightarrow \psi_1 \land \neg p \rightarrow \neg \psi_1 \land     \\
		q \rightarrow \gamma_1 \land \neg q \rightarrow \neg \gamma_1 \land \\
		(p \lor \neg p) \land (q \lor \neg q)
		\label{eq:7}
	\end{split}
\end{align}

When the night phase comes, agent \textit{a} will interrogate agent \textit{b},
resulting in agent \textit{a} knowing the faction of agent \textit{b}. Agent
\textit{a} will come to know that agent \textit{b} is a member of the mafia.
Directly proving $\neg p$, since the role of a villager belongs to the town
faction, not the mafia faction. This action also excludes all active world from
the sheriff's knowledge, where agent \textit{b} is a member of the town,
leaving him only with world wherein he is a member of the mafia, which in this
case mans he must be the godfather. During the same night, agent \textit{b}
targets, agent \textit{c}, killing him. The killing is then publicly announced
during the morning phase, along with the role of the victim. Recall from
\cref{eq:6}, we can say $\varphi_1!G\land\varphi_1$, which informally is that
everything before the public announcement, and the announcement itself, is now
applicable. In our simulation, this is done by calling
\lstinline[]{updatePossibleWorlds(information)}, displayed in
\ref{lst:updateWorld}

\begin{algorithm}[h]
	\caption{Snippet from appendix C}
	\begin{algorithmic}[1]
		\Function{updatePossWorlds}{information}
		\ForAll{p $in$ game.Ps}
		\ForAll{possWorld $in$ p.possWorlds}
		\State possWorld.UpdateMarks(information)
		\EndFor
		\EndFor
		\EndFunction
	\end{algorithmic}
\end{algorithm}\label{lst:updateWorld}

Which simply iterates over all players and all their respective worlds, and
updates the information. Now recall back to \cref{eq:7}, after the newly
acquired information of both $\neg \psi_1$, the knowledge that agent \textit{b}
is not a villager, and $\gamma_2$, the role of agent \textit{c} we can now
simplify the previous expression, which results in:
\begin{align}
	G' = K_a(\varphi \land \neg \psi_1 \land \gamma_1) \land \neg p \land q
\end{align}
Furthermore, as agent \textit{a} began with the knowledge of agent \textit{b}
being either a godfather, or a villager, and they now know that they are not a
villager, they can infer that they must be a godfather, since that is the only
world that has not been excluded, based on the knowledge they gained during the
night. \\
The game then continues with a new communication phase, and the game goes on,
for a short while longer, until the godfather has been voted out.
