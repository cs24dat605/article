\onecolumn
\begin{center}
	\section*{Summary}\label{sec:summary}
\end{center}
This paper investigates the disparity between win-rates of opposing teams in
social deduction games with an abundance of special roles. To do  this, an
approach based on epistemic logic and common sense is used to create a
simulation that emulates physical games of this type. The foundation of the
game is based on similar approaches within the same field, combining previous works based on dynamic epistemology with inquiry epistemology,
which we hypothesize to more realistically reflect the actual game. It allows agents to also model the anticipated answer to any question, and then build goal-oriented
paths of questions to other agents, from which they can deduce and infer facts or reveal lies. Additionally, agents are able to update their modelled view of worlds, based on newly acquired facts. \\
This combination of fields in epistemology lays the groundwork of inference in the simulation, 
as well as providing a formal framework for the description of behavior 
reliant on logical formulae within the simulation. \\
The simulation was built with a truth-table approach, resulting in a heavy
computational burden, but with the benefit of being easy to understand
intuitively. \\
The simulation was run with a multitude of different team compositions, which
resulted in the revelation that the most important factor to consider when
aiming for the creation of balanced social deduction games is that killing-heavy mafia teams perform better than any other compositions. Secondly, the ratio of mafia members to town members, is very impactful, but killing-heavy compositions may equalize this imbalance somewhat. Lastly, excluding killing-heavy teams showed that teams with the ability to acquire new information quickly performed better than those who could stop the town from performing their nightly actions. 
\twocolumn