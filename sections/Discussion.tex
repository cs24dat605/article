\section{Discussion}\label{sec:discussion}
One can ponder that these results may be influenced by the 
limited size of the teams used in the games, as the 
blackmailer and consort roles may 
thrive when being able to
use
their actions on powerful town roles. But with a mafia team 
of size 3 these roles may lack enough investigative and
murderous power, resulting in them being either unable to 
find the right targets, or finish the job if they do find 
them.\\
Our results have provided useful insights into the 
impactfulness of each mafia role, and with a different 
approach to the implementation, one might be able to 
construct larger game sizes without the heavy computational 
burden of this chosen approach. Our results also shows us, 
just how unpredictable these type of games are, and regardless of what composition we choose,
both sides still have a realistic probability of winning, in 
many cases, even close to equal chances.\\
Furthermore, the data shows that one thing that is even more 
important than the roles of the mafia members, is the ratio 
of mafia members in relation to the size of the town, as 
almost all compositions of mafia teams with 3 mafia members 
performed better than the mafia consisting of only 2 members. 