\section{Related Works} \label{sec:RelatedWorks}
A paper authored by Eger and Martens\cite{commitment} analysed the effects of
commitment to a plan in the game One Night Ultimate Werewolf. They modeled this
using Dynamic Epistemic Logic(DEL), allowing each agent to be aware of multiple
different possible worlds with alternating sets of facts, and sorting them
based on the least likelihood of it being true. This is achieved by ‘marking’
worlds that contradict statements made by other agents.

When an agent needs to choose what to base their next action on, they determine
that based on the number of marks of other players, or randomly, if they have
no facts. They proclaim this gives the agents a certain degree of planning.
However, due to the increasing amount of worlds generated by new actions
performed, many of them equivalent, a notion of commitment is introduced. This
means that they will stick with a certain plan until another plan is valued
significantly higher. They compare this commitment to a baseline of random
actions by villagers, resulting in a high win percentage for werewolves,
proving their hypothesis. In a game of both werewolves and villagers with
weighted worlds, they found that simply valuing the plans equally, that is
choosing the new plan if it is strictly better than the previous, resulted in
the lowest win percentage for werewolves. \todo{Speculate how this relates to
    what we do}\\ \\ To formally support a model where players can ask each other
questions, and also gain additional knowledge throughout the game, two
epistemological concepts must be introduced. The Interrogative Model of Inquiry
(IMI) and Dynamic Epistemic Logic are both central paradigms of formal
epistemology that support these features, and where one lacks, the other
excels. While IMI provides knowledge, reasoning, and inference in a multi-turn
information-gathering process between an Inquirer and Nature, DEL models how
each agent updates their knowledge in a multi-agent setting.

In 2014 Y.Hamami\cite{delimi} published a formalization of how to combine
these, which combines the information-gathering process in an inquiry with
respect to multiagent dimensions. As opposed to standard IMI or DEL, the
proposed DEL\textsubscript{IMI} sees information as acquired by either common
knowledge or from other agents in an interrogation. \\ The inclusion of agents
also introduces the notion of lying, that is answering an interrogative
question with non-factual information, closely representing the nature of
social deduction games. In reality, both the intrigue and the complexity of
social deduction games often arise from the discussions involved between
players, which we see as a core part of this type of game. While other works
related to social deduction games have resorted mostly to rely on simple
one-way actions, the DEL\textsubscript{IMI} model allows agents to interrogate
others, perhaps challenge their claims, find inconsistencies, or use logical
inference to establish presuppositions.\\ \\ To formally describe the concepts
related to the modelling of this simulation, we draw upon the foundations of
dynamic epistemic logic as well as formalisations for describing multi-agent
systems. Such sources are related to dynamic epistemic inquiry\cite{delimi},
modelling knowledge- and
information\cite{modelling_multi_agent_epistemic_systems}, planning with common
knowledge\cite{multi_agent_epistemic_planner_common_knowledge}, and
probabilistic approaches\cite{probibalistic_multiagent_systems}. All of these
were significant in the pursuit of knowledge regarding this area of research,
and in building an adequate vocabulary to describe actions, beliefs, knowledge,
and epistemology in general.