\section{Results}\label{sec:results}
The original rules for the mafia game excluded roles altogether, instead 
preferring the simplest version of the game consisting of two teams, the mafia 
and the town. Using these rules, it was recommended to have 1 mafia member per. 
3 players, leading to a game setup of 3 mafia members and 7 town 
members\cite{MafiaRules}. Using this game setup in our simulation leads to the 
mafia having a clear advantage, as can be seen on figure 
\ref{fig:VariousSimulations}. The is likely due to the expanded killing 
capabilities in the form of the mafioso and the godfather, relative to a 
limited and a expansion of the town capabilities, in the form of the sheriff 
and the doctor. \\
Should one instead reduce the number of mafia members to 2, and as a result 
increase the number of town members to 8, the win-rates for the two teams 
equalize significantly more, but still being to the mafias favour (MGSD \& 
MGS).\\
Getting to the subject of this report: How a mafia fares against a powerful 
town\footnote{A powerful town being defined as one that uses all roles 
from 
appendix \ref{app:A}.}, one can look at all of the columns after the "Powerful 
Town" label on the figure (\ref{fig:VariousSimulations}). The only consistent 
feature of the mafia teams represented by these columns are the presence of 1 
godfather on each team. They display that, against a powerful town, many 
configurations of mafia-roles are quite balanced, resulting in mostly a 50/50 
win-rate for each team. The main exceptions for this is in regards to 
killing-heavy compositions the most significant of which being MMG, with a 
win-rate of 74\% for the mafia. Furthermore, it seems that one of the only 
roles that can somewhat replace a mafioso in mixed-roles compositions in the 
consigliere, as such compositions (CgCtG \& CgBG) fare comparably to ones 
including a mafioso. \\
Below are all of the results:
\begin{figure}[H]
    \includegraphics[width=1\linewidth]{figures/Winrates}
    \caption{\\Graph of the win-rate of various simulated game compositions.\\
        There were 10 players in each simulation.\\
        Each simulation was 100 games.
        Simulation names are related to team composition, based on role
        abbreviations in appendix \ref{app:A}.\\
        The first 4 runs labelled \textit{Villagers} means that all roles not
        mentioned in the abbreviation are replaced with villagers.\\
        The next runs labelled \textit{Powerful Town} means that all roles not
        mentioned in the abbreviation are replaced with	one of each town role.}
    \label{fig:VariousSimulations}
\end{figure}
\vspace{-5px}Based on this graph we can deduce that mafiosi and
consiglieri are more
impactful for the mafia than both consorts and blackmailers. But it also seems
that the varied team consisting of one mafioso, consigliere, and godfather,
performs marginally better than other configurations. \\
The results indicate that when playing against a powerful town, it is more
important for the mafia to be able to continually kill town members, and to be
able to quickly find the most powerful roles of the town, in order to eliminate
them, than it is to deny actions and communication to the town. \\
It would be interesting to see which of the non-killing mafia roles perform
best when they are on their own, and which composition of them perform best.
This type of game removes the mafias ability to kill during the night, while
they retain their knowledge of one another and their respective abilities to
gain information, limit nightly actions, and limit communicative actions. Below
are the results of such games:
\begin{figure}[H]
    \includegraphics[width=1\linewidth]{figures/Winrates_NonKilling}
    \caption{\\Graph of the win-rate of various simulated game compositions
        with only non-killing mafia roles.\\
        There were 10 players in each simulation.\\
        Each simulation was 100 games.
        Simulation names are related to team composition, based on role
        abbreviations in appendix \ref{app:A}.\\
        All of the runs were against the previously explained powerful town.}
    \label{fig:VariousSimulationsNonKilling}
\end{figure}
\vspace{-5px} As seen on the above graph we can once again conclude that
consiglieri are more impactful than either consorts or blackmailers \ref{fig:VariousSimulationsNonKilling}. This may
be due to their superior abilities in finding the sheriff, enabling them to
collectively vote them out. It also seems that no mixed-combination of
non-killing mafia roles can compete with the pure-consiglieri team.
