\section{Results}\label{sec:results}
To further validate our model we ran a test with the recommended setup of 3 
mafia members and 7 town members\footnote[1]{1 Godfather, 2 Mafioso, 1 Sheriff, 
1 
Doctor, 5 Villagers}. This yielded a win-rate for each team of 50\%. Since this 
balanced game was achieved by following the recommended rules, it seemed that 
our model indeed portrayed real games with some level of accuracy. We then ran 
experiments with a team biased toward the town\footnote{Using one of each 
town 
role from appendix \ref{app:A}, against 1 Godfather, and 2 Mafioso}. This 
yielded a win-rate of 75\% favoring the town. This was expected, and further 
validating 
the purpose of this paper: How can one re-balance the win-rates when facing a 
powerful town. We then simulated games with every possible combination of mafia 
roles, against this baseline of a town consisting of one of each role. The only 
consistent role on the mafia team was the presence of 1 Godfather. Below are 
the results: \\
\begin{figure}[h]
	\includegraphics[width=1\linewidth]{figures/placeholder}
	\caption{Graph of the win-rate of various simulated game compositions.}
	\label{fig:placeholder}
\end{figure}
\\Based on this graph we can deduce that ... are more impactful for the mafia 
that both ... and .... But it also seems that the varied team consisting of 
..., ..., and ... yields the best results. However, it can also be seen, that 
when dealing with such a powerful town, not much can really be done in regards 
to win-rate, when constrained to the recommended 3 mafia members in a 10 player 
game. We could therefore examine a new baseline, of 4 mafia members, against 6 
townspeople, removing the lone villager for the town. This would yield a 
baseline of ...\% favoring the town, much less than before, but it gives us a 
little more wiggle-room to work with.