\section{Results}\label{sec:results}
The simulations were run with various 
role-compositions. The specific compositions 
can be decoded by using the label for each 
column and referring to the abbreviations for 
roles in appendix \ref{app:A}. The first five 
columns in figure 
\ref{fig:VariousSimulations} depicts runs 
where all undefined roles are defined to be 
villagers. In the following columns, after 
the label "Powerful Town", the undefined 
roles are determined to be one of each town 
role from appendix \ref{app:A}. There were 10 
agents in each simulation, and each 
simulation was run 100 times.\\\\
The original rules for the mafia game 
excluded roles altogether, instead
preferring the simplest version of the game consisting of two teams, the mafia
and the town. Using these rules\cite{MafiaRules}, it was recommended to have 1
mafia member per. 3 players, leading to a game setup of 3 mafia members and 7
town members. The use of this game setup in our simulation leads to the mafia
having a clear advantage, as can be seen on figure \ref{fig:VariousSimulations}
(MMGS \& MMGSD)\footnote{MMGS Refers to Mafioso, Mafioso, Godfather, Sheriff,
    and Doctor, as a role abbreviation, others can be found in \ref{app:roles}}.
The is likely due to the expanded elimination capabilities in the form of the
mafioso and the godfather. \\ Should one instead reduce the number of mafia
members to 2, and as a result increase the number of town members to 8, the
win-rates for the two teams equalize significantly more, but still being to the
mafias favour (MGSD \& MGS).\\ Getting to the subject of this report: How a
mafia fares against a powerful town\footnote{A powerful town being defined as
    one that uses all roles from appendix \ref{app:A}.}, one can look at all of the
columns after the "Powerful Town" label on the figure
(\ref{fig:VariousSimulations}). The only consistent feature of the mafia teams
represented by these columns are the presence of 1 godfather on each team. They
display that, against a powerful town, many configurations of mafia-roles are
quite balanced, resulting in mostly a 50/50 win-rate for each team. The main
exceptions for this is elimination-heavy compositions the most significant of
which being MMG, with a win-rate of 74\% for the mafia. Furthermore, it seems
that one of the only roles that can somewhat replace a mafioso in mixed-roles
compositions is the consigliere, as such compositions (CgCtG \& CgBG) fare
comparably to ones including a mafioso. \\ Below are all of the results:
\begin{figure}[H]
    \includegraphics[width=1\linewidth]{figures/Winrates}
    \caption{Graph of the win-rate of various 
    simulated game compositions.}
    \label{fig:VariousSimulations}
\end{figure}
\vspace{-5px}Based on this graph we can deduce that mafiosi and
consiglieri are more
impactful for the mafia than both consorts and blackmailers. But it also seems
that the varied team consisting of one mafioso, consigliere, and godfather,
performs marginally better than other mixed configurations. \\
The results indicate that when playing 
against a powerful town, it is most
important for the mafia to be able to 
continually eliminate town members. This can 
be seen in column MMG on figure 
\ref{fig:VariousSimulations}, as this 
role-composition is the most successful for 
the mafia, and the one able to continue 
eliminating town-members for the longest.\\
It would be interesting to see which of the non-eliminating mafia roles perform
best when they are on their own, and which composition of them performs best.
This type of game removes the mafias ability to eliminate during the night, while
they retain their knowledge of one another and their respective abilities to
gain information, limit nightly actions, and limit communicative actions. Figure \ref{fig:VariousSimulationsNonKilling} shows
are the results of such games:
\begin{figure}[H]
    \includegraphics[width=1\linewidth]{figures/Winrates_NonKilling}
    \caption{Graph of the win-rate of various 
    simulated game compositions with only 
    non-eliminating mafia roles.\\
        All of the runs were against the previously explained powerful town.}
    \label{fig:VariousSimulationsNonKilling}
\end{figure}
\vspace{-5px} As seen on the above graph we can once again conclude that
consiglieri are more impactful than either consorts or blackmailers \ref{fig:VariousSimulationsNonKilling}. This may
be due to their superior abilities in finding the sheriff, enabling them to
collectively vote them out. It also seems that no mixed-combination of
non-killing mafia roles can compete with the pure-consiglieri team, except for BBCg.
