\section{Notational syntax}
The terminology and its notation used throughout this article will closely
align with \cite{delimi}, although not all aspects will be presented here. This
section will cover what is necessary to reach our dynamic inquiry epistemic
logic model. Recall that in \textbf{DEL}s, factual information is represented
in the propositional language and is therefore also the inquiry language of the
oracle $\mathscr{P}$: $$ \gamma::= p\:|\:\neg\gamma|(\gamma\land\gamma) $$

where \textbf{P} is a countable set of atomic propositions (i.e facts) and $p
	\in \mathbf{P}$. This reads as a proposition is either a fact, the negation of
a proposition or the conjunction of multiple propositions.

By the previous definition, we can now describe the \textit{static IMI
	epistemic language} $\mathscr{L_s}$, which is defined as:
\begin{align*}
	\varphi ::= p |\neg\varphi|(\varphi \land \varphi) | K_a\varphi | A_a\varphi | \phi\gamma \\| R_\gamma((\gamma_1,...,\gamma_k,), \gamma_i) | R_a((\varphi_1,...,\varphi_k), \varphi_j) \\ \text{where p} \in \text{\textbf{P}}\text{, a} \in \text{\textbf{Ag}}
\end{align*}
where \textbf{Ag} is a set of agents, the \textbf{knowledge operator} $K_a\varphi$ reads "agent \textit{a} implicitly knows that $\varphi$", the \textbf{awareness operator} $A_a\varphi$ reads as "agent \textit{a} is aware of $\varphi$", the formula $\phi\gamma$ reads as $\gamma$ is in the answer set of the oracle. $R_\gamma((\gamma_1,...,\gamma_k,), \gamma_i)$ reads as "$\gamma_i$ is the answer the oracle will provide to question $\gamma_1,...,\gamma_k$", similarly $R_a$ denotes what answer agent \textit{a} will answer to a question.

