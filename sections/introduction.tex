\section{Introduction}
When trying to solve a problem, many solutions may present themselves. However, 
most of these  will come with unintended side effects. As social deductions 
games have become increasingly popular, one problem that has arisen is that 
such games often contain one specific role with no special actions aside from 
the fact that it will be the most numerous role in the game. This results in 
the majority of players being assigned a role which can be considered boring. A 
way to solve this problem, is to assign more fun or interesting roles to those 
people, but that comes with the unintended side effect of affecting the 
win-rate of the different teams. This paper will look into mitigating this side 
effect by adding more interesting roles to the opposing team as well, to see 
what roles have the most impact on the win-rate of both teams. The roles that 
may be added consist mainly of two types: Information Gatherers and Information 
Deniers. Through this analysis we will be able to draw general conclusions on 
the topic of whether it is more important to gather information for your own 
benefit, or to deny information to opposing entities. 

with man
variants being released in recent years. These games typically include
different roles, actions and intense discussions adding complexity to an
otherwise simple game premise.

These games pit players against each other through lies and deceit, tempting
players into manipulating others with the intent to further their own goals. In
general, these games have two factions, or teams, with players being given
roles that may belong to either team. These roles may be simple, having no
additional ‘special’ actions, or they may be able to impact the course of the
game, and the gathering of information, in unique ways. Furthermore, these
games typically have different, competing, win conditions for each faction.

One such game is the online video game ‘Town of Salem’. It is a social
deduction game based on the game ‘Werewolf’, which itself is inspired by the
original version of the game ‘Mafia’. These three games share the similarities,
that there are typically two unequally sized teams, the town, being the larger,
and the werewolves/mafia, being the smaller. The smaller team will henceforth
be referred to as the Mafia. The games also share that they consist of four
phases, the night phase where special actions take place, the morning phase
where the public results of the night phase are revealed, the day phase where
people discuss based on their acquired information, and the voting phase where
everyone on both teams vote on who to lynch during the day phase. Then, the
game repeats with a new night phase.

The base game has two factions; the town and the mafia. The town wins by
collectively lynching all of the mafia members, and the mafia wins by achieving
a numerical player majority within the game. The roles most often used in these
games include: An uninformed, town-allied majority with no special powers. The
mafia, which will slowly eliminate the majority during the multiple night
phases, and attempt to secretly spread misinformation throughout the town.
Investigative roles, allied to the town, which have different methods for
acquiring facts, such as privately viewing another player’s team during the
night phase.

One thing that is often overlooked in these games, is the simple fact, that in
a game with multiple roles, being assigned one with no special actions, can
feel boring, making the assigned player feel disinterested in the game. One
method used to remedy this, is to add additional roles to the town team, since
this team mostly consists of uninformed players with no special abilities.

A side-effect of this is that the win rate of the town increases, as the
addition of these roles necessarily result in more information being acquired
by the town team. What we would like to investigate in this paper is which
roles could be added to the mafia team in order to equalize this growing unfair
win rate. We would like to see which mafia roles have the highest impact on win
rate, based on the number of initial players, the ratio of teams, different
play-styles, and the starting composition of the town team.

This will be analyzed by utilizing a self-developed, console-based simulation
of the game, utilizing a probabilistic approach to decision making, with a
foundation in epistemic logic. The game will be modeled with different role
actions, voting strategies, communicative actions, and levels of selflessness.

A sizable amount of research has been put into the field of social deduction
games. However, much of that research has been focused on strategies relating
to when and how to utilize different communicative actions\cite{commitment}.
Others focus on ways for the town to detect the mafia
members\cite{werewolf_stealth}, or protect key town members from the
mafia\cite{werewolf_nash_equilibrium}, and others still focus on the different
decision making strategies that can be implemented in multi agent systems,
which these games can be modeled
as\cite{modelling_multi_agent_epistemic_systems}\cite{multi_agent_epistemic_planner_common_knowledge}\cite{probibalistic_multiagent_systems}.

However, one thing that is noticeably absent from this research is analyses
with the focus of "evening the odds" for the mafia team when faced with a town
full of special roles. The outcome of such an analysis might be a game setup
where everyone involved can have a role with special abilities, meaning that
players will likely be more engaged and active, throughout all stages of the
game.