\section{Introduction}
One pastime that has become increasingly popular in recent years is playing
games with elements of social deduction. These games, be they video games or
board games, pit players against each other through lies and deceit, tempting
players into manipulating others with the intent to further their own goals. In
general, these games have two factions, or teams, with players being given
roles that may belong to either team. These roles may be simple, having no
additional ‘special’ actions, or they may be able to impact the course of the
game, and the gathering of information, in unique ways. Furthermore, these
games typically have different, competing win conditions for each faction,
which are known to the group of players. \\ One such game is the video game ‘Town
of Salem’. It is a social deduction game based on the game ‘Werewolf’, which
itself is inspired by the original version of the game ‘Mafia’. These three
games share the similarities, that there are typically two unequally sized
teams, the town, being the larger, and the werewolves/mafia, being the smaller,
henceforth referred to as only the Mafia. They also share that the game
consists of four phases, the night phase where special actions take place, the
morning phase where the public results of the night phase are revealed, the day
phase where people discuss based on their acquired information, and the voting
phase where everyone on both teams vote on who to lynch during this day phase.
Then, the game repeats with a new night phase. \\ The base game has two factions;
the town and the mafia. The town wins by collectively lynching all of the mafia
members, and the mafia wins by achieving a numerical player majority within the
game. The roles most often used in these games include: An uninformed,
town-allied majority with no special powers. The mafia, which will slowly
eliminate the majority during the multiple night phases, and attempt to
secretly spread misinformation throughout the town. Investigative roles, allied
to the town, which have different methods for acquiring facts, such as
privately viewing another player’s team during the night phase. In these games,
there are many different parameters to tweak, to change the probabilistic
outcome of the players’ actions. \\ This paper will look at the collective
framework that these three games provide, to have access to as many of these
parameters as possible. Changeable parameters include; the number of players in
the game, the ratio between the teams, the introduction of a neutral faction,
the different playstyles of the players, whether the other players are aware of
these playstyles, and lastly the introduction of different roles for all teams
in the game. We would like to investigate the impact that different roles may
have on the games, when utilizing a probabilistic approach to decision making,
with a foundation in epistemic logic. The game will be modeled with different
role actions, voting strategies, communicatory actions, and levels of
selflessness.