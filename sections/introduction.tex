\section{Introduction}
When trying to solve a problem, many solutions may present themselves. However, 
most of these will come with unintended side effects. In the world of social 
deduction games, one specific problem that has presented itself is the fact 
that as more diverse roles has been introduced, in order to keep the games 
fresh, new, and exciting, an imbalance has presented itself in the win-rate 
between the two teams. \\ \\
Most social deduction games have the same foundation: Two teams, one larger 
with uninformed players, and one smaller with informed players. Both teams want 
to eliminate the other team, with differing methods of accomplishing this. Often 
times, the uninformed team have will have a small number of special roles 
assigned to it, in order to be able to acquire information, to be able to root 
out the smaller team. The smaller team typically tries to stay hidden, while 
eliminating the players in the bigger team.\\ \\
Historically, the larger team has typically been weighed down with many players 
being assigned low-impact roles. 
As games have evolved, players have grown bored of this dynamic and has led to 
the introduction of more roles, in order to excite the players who were 
previously assigned the low-impact roles. But as their roles are no longer 
low-impact, this has skewed the win-rate of their team in their favor. This is 
problematic from a balance perspective, and therefore this paper will look at 
how one can analyze and introduce roles to the opposing team, in order to 
equalize this imbalance. The roles that will be analyzed consist mainly of two 
types: Information Gatherers and Information Deniers. Through this analysis it 
will be able to draw general conclusions on the topic of whether it is more 
important to gather information for your own benefit, or to deny information to 
opposing entities. \\

This will be analyzed by utilizing a self-developed, console-based simulation
of the game, utilizing a probabilistic approach to decision making, with a
foundation in epistemic logic. The game will be modeled with a variety of roles and actions, as well as voting strategies and communicative options. 

A sizable amount of research has been put into the field of social deduction
games. However, much of that research has been focused on strategies relating
to when and how to utilize different communicative actions\cite{commitment}.
Others focus on ways for the town to detect the mafia
members\cite{werewolf_stealth}, or protect key town members from the
mafia\cite{werewolf_nash_equilibrium}, and others still focus on the different
decision making strategies that can be implemented in multi agent systems,
which these games can be modeled
as\cite{modelling_multi_agent_epistemic_systems}\cite{multi_agent_epistemic_planner_common_knowledge}\cite{probibalistic_multiagent_systems}.

However, one thing that is noticeably absent from this research is analyses
with the focus of mitigating the previously mentioned side-effect, that arises 
when playing against a team consisting of mostly high-impact roles. 
Additionally, there is minimal research on the impact of the ability to inquire 
other players, regarding their knowledge of the world. The outcome of such an 
analysis might be a game setup where everyone involved can have a role with 
special abilities. This may also mean that players will be more engaged and 
active throughout all stages of the game. Lastly, it may also grant some 
insight into whether it is more beneficial to gather information for oneself, 
or to deny information to others. 