\section{Introduction}
When trying to solve a problem, many solutions may present themselves. However,
most of these will come with unintended side effects. In the world of social
deduction games, one specific problem has manifested itself. As more diverse
roles has been introduced, in order to keep the games fresh, new, and exciting,
an imbalance has presented itself in the win-rate between the opposing teams.
\\ \\ Most social deduction games have the same foundation: The players are
divided into two teams, one larger with uninformed players, and one smaller
with informed players. Each of the teams wants to eliminate the other team,
with differing methods of accomplishing this. Often, the uninformed team will
have a few special roles assigned to it, in order to be able to acquire
information, to be able to root out the smaller team. The smaller team
typically tries to stay hidden, while eliminating the players in the bigger
team.\\ \\ Historically, the larger team has typically been burdened by the
assignment of many low-impact roles to its players. As games have evolved,
players have grown bored with this dynamic and that has led to the introduction
of more roles, in order to excite the players who were previously assigned the
low-impact roles. But as their roles are no longer low-impact, this has skewed
the win-rate of their team in their favour. This is problematic from a balance
perspective, and therefore this paper will look at how one can analyse and
introduce roles to the opposing team, in order to equalize this imbalance. The
roles that will be analysed consist mainly of two types: information- gatherers
and deniers. Through this analysis we will be able to draw a general conclusion
on the topic of whether it is more important to gather information for your own
benefit, or to deny information to opposing entities.

This will be analysed by utilizing a self-developed, console-based simulation
of a social deduction game, incorporating a numeric approach to
decision-making, with a foundation in epistemic logic. The game will be
modelled with a variety of roles and actions, as well as voting strategies and
communicative actions.

A sizeable amount of research has been put into the field of social deduction
games. However, much of that research has been focused on strategies relating
to when and how to utilize different communicative actions\cite{commitment}.
Some focus on ways for the town to detect the mafia
members\cite{werewolf_stealth}, or protect key town members from the
mafia\cite{werewolf_nash_equilibrium}. Others focus on the different
decision-making strategies that can be implemented in multiagent systems, which
these games can be modelled
as\cite{modelling_multi_agent_epistemic_systems}\cite{multi_agent_epistemic_planner_common_knowledge}\cite{probibalistic_multiagent_systems}.

However, one thing that is noticeably absent from this research is analyses
with the focus of mitigating the previously mentioned side effect, that arises
when playing against a team consisting of mostly high-impact roles.
Additionally, there is minimal research on the ability to inquire other
players, regarding their knowledge of the world. The outcome of such an
analysis might be a game setup where everyone involved can have a role with
special abilities. This may also mean that players will be more engaged and
active throughout all stages of the game. Lastly, it may also grant some
insight into whether it is more beneficial to gather information for oneself,
or to deny information to others.\\ \\

In the following section \ref{sec:RelatedWorks}, we will review papers and
studies that have significantly contributed to our work.

We then describe how we intend to model our social deduction game in section
\ref{sec:Modelling}. It outlines the rules of the game and its operational
mechanics. Key roles will be explained in detail, as well as how agents
communicate with each other. Furthermore, we will explain our 'mark' system and
'world elimination' in this section.

Moving over to the dynamic inquire language section
\ref{sec:DynamicInquiryLanguage}. This section begins with an introduction to
the terminology and the inquiry language. We then expand this to a static
inquiry language with agents and knowledge. Finally, we conclude the section by
expanding it to a dynamic inquiry language with model updates.

For the implementation section \ref{sec:implementation}, we provide a detailed
walkthrough of all the processes utilized in our implementation. We start by
introducing the various data structures and then explain how each 'phase' of
the game is computed. \todo{Adjust to follow the new format when section 5 is
    updated}

In the final sections of the paper \ref{sec:results}, \ref{sec:discussion}, and
\ref{sec:conclusion}, we present all our findings from our experiments.