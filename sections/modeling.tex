\section{Modeling}
As mentioned earlier, we wish to model a framework of games instead of one
specific one, in order to encapsulate as many parameters as possible. The game
that will be modeled will have two factions, henceforth known as the town and
mafia. The town wins when they have eliminated all mafia members, and the mafia
wins when their factions have the non-strict numerical majority of players. The
game is played out in rounds consisting of four phases, which happen in order,
until either win-condition is met.

The night phase is where different roles perform private or public actions to
gain information or spread misinformation. The morning phase is where public
results of the night phase are revealed to all players. The role of players who
died during the night phase is revealed at this time. The day phase is where
players try to influence each other's worldviews through communicative actions.
The voting phase is where players vote for who to lynch according to their
individual worldviews. The voting phase can end in a majority vote which will
result in a lynching of that player, killing them and revealing their role
publicly, or in a tie, which will result in the phase ending with no lynching.

There are many possible roles for players to be assigned. In typical games, the
most numerous one would be the villager who has no special powers. However, as
this paper wishes to illuminate the way in which the mafia can even the odds
against a town populated by special roles, they will not be used extensively
here. All roles are explained in appendix \ref{app:A}, but the ones allied to
the mafia will be gone over here, to explain how they work, in order to give
the reader an intuition as to which might be the most impactful.

\begin{enumerate}
	\item\textbf{Godfather} must choose one player each night who will die.\footnote{The only scenario except by vote in which the Godfather dies, is if they target the Veteran, and the Mafioso is targeted by the escort on the same night.}
	\item\textbf{Mafioso} performs the killing on the Godfathers behalf.
	\item\textbf{Consigliere} must choose one player each night and privately
	      gain information regarding their role.
	\item\textbf{Consort} must choose one player each night and prevent them
	      from performing any actions during the night phase.
	\item\textbf{Framer} must choose one player each night to frame. A framed
	      player appears to be a member of the mafia if chosen by the Sheriff on this
	      night.
	\item\textbf{Blackmailer} must choose one player each night to blackmail. A
	      blackmailed player can only choose the “do not say anything” communicative
	      action.
\end{enumerate}

Now, to simulate the games as closely as possible to the real world we use a
foundation built upon a probabilistic approach to decision making, and an
epistemological approach to the acquisition and refreshing of
knowledge\cite{commitment}. Their core principles are that players keep track
of all possible worlds that may be possible based on their own knowledge, and
the public knowledge available to them. They keep track of beliefs and lies by
“marking” worlds which contradict statements made by other players. Due to the
fact that there should always be more players belonging to the town than the
mafia, and that the players of the town have no reason to lie, it is presumed
that the world with fewest “marks” must be the true world, which the player
will base their decisions on. A few modifications to their methodology have
been made, in order to better suit a multi-round game.

First, the communicative actions for the game have been altered slightly. In
this simulation, players can choose between 6 different communicative actions:
\begin{enumerate}
	\item Claim to have a certain role.
	\item Claim that someone else has a certain role.
	\item Claim to have performed a nightly actions, with some result.
	\item Inquire another player regarding their role.
	\item Inquire another player regarding their beliefs about other players.
	\item Do not say anything.
\end{enumerate}
The purpose of these actions are to mimic the in-person game as closely as
possible. Players will each be able to take a variable amount of communicative actions before the
phase concludes. Players choose which actions to take, based on which action
will influence others’ most towards their own world view. If multiple actions
are in accordance with their world view, then they may choose to inquire other
players in order to maximize their information gain.

When public information is revealed, like the role of a killed player, all
players must update their worldviews, eliminating all worlds not in accordance
with this information. This will slowly limit the possible worlds, granting
more information to the town. The marks generated by communicative actions are
associated with the player that generated it, meaning that whenever public
information is revealed about that player, villagers may either discard or
reinforce said marks, depending on whether the player was revealed to be a
member of the mafia, meaning that may have had incentive to lie, or if they
were a villager, meaning they should have only told the truth.

Another thing, kept track of by the players, are the active and inactive
worlds. An active world is a world that is true based on the public and private
information available to a player. An inactive world is a world that is true
based on the public information available, but false based on the private
information available. The need for inactive worlds, is that a player cannot
assume that other players are aware of private information, so these worlds
must be included when deciding which communicative actions to perform during
the day phase. An example:

Player 1 is a sheriff (a role allied to the town, who functions similarly to
the consigliere), and has used their nightly action to look at the faction of
player 2, which gave them the private information that player 2's faction is of
the town. Player 1 can now mark all worlds where player 2 is part of the mafia
as inactive, as they are factually false, but since this is not public
information player 1 knows that not everyone knows this, and these worlds must
therefore still be considered when choosing communicative actions, even if just
for knowing which worlds to avoid promoting.
